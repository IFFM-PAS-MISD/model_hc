% This is samplepaper.tex, a sample chapter demonstrating the
% LLNCS macro package for Springer Computer Science proceedings;
% Version 2.20 of 2017/10/04
%
\documentclass[runningheads]{llncs}
%
\usepackage[top=5cm, bottom=5.6cm, left=4.5cm, right=4.2cm]{geometry}
\usepackage{graphicx}
\usepackage{array}
\newcolumntype{P}[1]{>{\centering\arraybackslash}p{#1}}
% Used for displaying a sample figure. If possible, figure files should
% be included in EPS format.
%
% If you use the hyperref package, please uncomment the following line
% to display URLs in blue roman font according to Springer's eBook style:
% \renewcommand\UrlFont{\color{blue}\rmfamily}

\makeatletter
\renewcommand\paragraph{\@startsection{paragraph}{4}{\z@}%
                                    {3.25ex \@plus1ex \@minus.2ex}%
                                    {-1em}%
                                    {\normalfont\normalsize\bfseries}}
\makeatother

\begin{document}
%
\title{Model--assisted approach for quantification of the severity of debonding in sandwich composites}
%
%\titlerunning{Abbreviated paper title}
% If the paper title is too long for the running head, you can set
% an abbreviated paper title here
%
\author{Piotr Fiborek\inst{1}\orcidID{0000-0002-5030-3312} \and
Pawe\l{} Kudela\inst{1}\orcidID{0000-0002-5130-6443}}
%\and Third Author\inst{3}\orcidID{2222--3333-4444-5555}}
%
\authorrunning{P. Fiborek and P. Kudela}
% First names are abbreviated in the running head.
% If there are more than two authors, 'et al.' is used.
%
\institute{Institute of Fluid Flow Machinery, Polish Academy of Sciences
\email{pfiborek@imp.gda.pl}}
%
\maketitle              % typeset the header of the contribution
%
%\begin{abstract}
\paragraph{Abstract.}
The methods of damage detection and localization have been developing for many decades, while the damage severity assessment is still in the preliminary stage.  Therefore, in this paper, an accurate numerical model of guided waves propagation is utilized to determine a function which relates damage size and the amplitude of the transmitted wave.

The time--domain spectral element method was adopted for simulation of guided waves propagating in honeycomb sandwich panel. The assumed damage was a circular debonding between the skin and the core, modelled by disconnection of the coinciding nodes within the defect area. To increase computation efficiency, two--dimensional spectral elements were used for modelling of the panel substructures. However, due to the low diameter to thickness ratio, three--dimensional spectral elements were used for modelling of piezoelectric transducers. Instead of calculating the effective material properties of the honeycomb structure by the homogenization process, the real geometry of the core was preserved to obtain accurate results. Consequently, resonant effects related to the size of the honeycomb cell affecting the wave propagation are noticeable. Displacements coupling between the adjacent substructures was guaranteed by the non--matching grids interfaced based on Lagrange multipliers. Moreover, the model was optimized for parallel computation on the graphics processing unit.

In the results of parametric studies, the relation between debonding size and the amplitude of the transmitted waves was determined, which will be used for damage severity assessment in the real structure.

\keywords{damage severity assessment \and honeycomb sandwich composites \and spectral element method \and guided waves.}
%\end{abstract}
\end{document}
%