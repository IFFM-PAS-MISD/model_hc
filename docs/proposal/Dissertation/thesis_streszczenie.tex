
\documentclass[12pt,a4paper,final]{article}
\usepackage[polish]{babel}
\usepackage[OT4]{fontenc}
\usepackage[top=3.5cm, bottom=2.54cm, left=3.5cm, right=2.0cm]{geometry}
\usepackage{graphicx}
\usepackage{array}
\newcolumntype{P}[1]{>{\centering\arraybackslash}p{#1}}
% Used for displaying a sample figure. If possible, figure files should
% be included in EPS format.
%
% If you use the hyperref package, please uncomment the following line
% to display URLs in blue roman font according to Springer's eBook style:
% \renewcommand\UrlFont{\color{blue}\rmfamily}

\title{\textbf{Modelowanie płyt przekładkowych i przetworników piezoelektrycznych celem identyfikacji wielkości uszkodzeń mechanicznych}}
\vspace{0.5cm}
\author{\textit{mgr inż. Piotr Fiborek}\\
		\multicolumn{1}{p{.7\textwidth}}{\centering
			\vspace{0.5cm}
		Instytut Maszyn Przepływowych\\
			im. Roberta Szewlaskiego\\
		Polskiej Akademii Nauk\\
	\vspace{0.5cm}
	Promotor: \textit{dr hab. inż. Pawe\l{} Kudela}\\
			}}

\begin{document}
\maketitle
%
\begin{abstract}

Celem pracy było opracowanie wspomaganej modelem metody oceny uszkodzeń mechanicznych w~płycie warstwowej z~rdzeniem o~strukturze plastra miodu. W~tym~celu zastosowano technikę monitorowania stanu technicznego konstrukcji opartą na propagacji fal prowadzonych. Badania umożliwiły nowe spojrzenie na~modelowanie propagacji fal w~złożonej strukturze w celu określenia, jak~rozmiar uszkodzenia wpływa na~charakterystyczne parametry propagacji.

Funkcja wpływu wielkości uszkodzenia na propagację fali mogłaby być uzyskana poprzez badania eksperymentalne lub analizę teoretyczną, ale obie metody podlegają pewnym ograniczeniom. Badania eksperymentalne wymagałyby wielu kosztownych próbek. Analiza teoretyczna, z drugiej strony, jest ograniczona do podstawowych elementów konstrukcyjnych (takich jak płyty, pręty i belki) w określonych warunkach brzegowych. Natomiast narzędzia modelowania numerycznego dostarczają dokładnych danych, oraz znajdują zastosowanie w~złożonych strukturach inżynierskich bez~nadwyrężania budżetu badań.

Analiza numeryczna propagacji fali kierowanej jest bardzo czasochłonna i~wymaga znacznych zasobów pamięci operacyjnej. Dlatego zastosowano metodę elementów spektralnych w dziedzinie czasu, która jest jedną z najdokładniejszych i~najbardziej wydajnych technik modelowania propagacji fal sprężystych. Algorytm całkowania w~czasie został zwektoryzowany w~celu prowadzenia obliczeń równoległych na~wielordzeniowej karcie graficznej. Zwiększono również efektywność obliczeniową poprzez redukcję globalnych stopni swobody stosując elementy dwuwymiarowe.
Podejścia do wyznaczania właściwości materiałowych kompozytów opierają się głównie na~procesie homogenizacji. Metoda ta~jest jednak niewystarczająca do~modelowania propagacji fal w~materiałach o~złożonej strukturze, takich jak~kompozyty warstwowe o~strukturze plastra miodu. Dlatego w~proponowanych badaniach opracowano model struktury materiału o~pełnej geometrii rdzenia. Struktura plastra miodu została zamodelowana za~pomocą powłokowych elementów spektralnych dla~każdej ściany komórki rdzenia.

W~analizie zastosowano podejście wielofizyczne, w~którym uwzględniono sprzężenie elektromechaniczne odpowiadające zastosowaniu przetworników piezoelektrycznych do~wzbudzania i~rejestrowania sygnału. Model został opracowany z~u\-względnieniem wpływu temperatury otoczenia. Dodatkowo, celem połączenia wszystkich komponentów struktury, zaimplementowano elementy interfejsowe gwarantujące ciągłość przemieszczeń sąsiednich elementów. Do~opracowania interfejsu zastosowano nowe podejście oparte na~mnożnikach Lagrange'a wykorzystujące funkcje kształtu elementów spektralnych do~połączenia dwóch niepasujących siatek.

W~celu walidacji modelu przeprowadzono pomiary eksperymentalne z wykorzystaniem (i)~analizatora impedancji do przetworników piezoelektrycznych, (ii) skanującego wibrometru laserowego Dopplera do~analizy pełnego pola propagacji fali oraz~(iii) piezoelektrycznego zestawu akwizycji fal do~punktowej analizy propagacji fal w~strukturze.

Przegląd literatury przedmiotu został przedstawiony na~początku rozprawy, gdzie opisano również metodologię, która została przyjęta do~realizacji założonego celu pracy. Następnie przedstawiono szczegółowy opis realizacji modelu propagacji fali w~strukturze warstwowej o~strukturze plastra miodu z~wykorzystaniem metody elementów spektralnych.
Po walidacji modelu przedstawiono analizę numeryczną mającą na~celu wyznaczenie funkcji identyfikacji uszkodzeń wspomaganej modelem z~uwzględnieniem zmiennej temperatury otoczenia. Następnie przeprowadzono badania parametryczne w~celu przedstawienia wpływu współczyników materiałowych płyty z~rdzeniem o~strukturze plastra miodu na~propagację fali. Na~koniec podsumowano rozprawę, przedstawiając wnioski wynikające z~przeprowadzonych analiz. 

Nowatorskość przeprowadzonych badań polegała na~opracowaniu nowego podejścia do~oceny stopnia uszkodzenia oraz~modelu numerycznego z~dokładną geometrią rdzenia o~strukturze plastra miodu wykorzystując metodę elementów spektralnych w~warunkach zmiennej temperatury otoczenia. Opracowana metoda może stanowić praktyczne kompendium wiedzy dla~projektantów systemów monitorowania stanu technicznego konstrukcji. Ponadto, w~ramach metody elementów spektralnych, opracowano interfejs do~połączenia dwóch sąsiednich elementów o~niepasujących siatkach, co~czyni tę~technikę bardziej elastyczną przy modelowaniu złożonych struktur.

%\keywords{damage severity assessment \and honeycomb sandwich composites \and spectral element method \and guided waves.}
\end{abstract}
\end{document}
%