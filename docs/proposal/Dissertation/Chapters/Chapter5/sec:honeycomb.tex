%% SECTION HEADER /////////////////////////////////////////////////////////////////////////////////////
\section{Mesh generation for the \acl{fcgm}}
\label{sec:honeycomb}

%% SECTION CONTENT ////////////////////////////////////////////////////////////////////////////////////
The modelled structure was composed of the following components: 2D for the core, epoxy adhesive and cyanoacrylate glue and 3D for the \ac{cfrp} plate and the \acp{pzt}.
Figure~\ref{fig:struct_mesh} depicts the spectral element used to model the wall, the skin and the \ac{pzt}.
During the creation of the core mesh, special attention was taken to minimise the number of non-zero values in the matrix \(\textbf{G}\).
\begin{figure}[H]
	\begin{center}
		\includegraphics[width=0.95\textwidth]{Chapter_5/struct_mesh}
	\end{center}
	\caption{The mesh with the nodes distribution, (\textbf{a}) spectral element used for modeling the wall of the core, (\textbf{b}) excerpt of the skin plate and (\textbf{c}) cyanoacrylate glue mesh with the second-order curve at the boundary}
	\label{fig:struct_mesh}
\end{figure}

The core elements were selected for the slave mesh, with one spectral element dedicated to each honeycomb cell wall.
The master meshes of the skin panel and adhesive layer were divided by three rhombic elements into the area under the core cell.
This way, the interface nodes coincided with those on the hexagon edges (red line in Figure~\ref{fig:struct_mesh}(\textbf{b})).
The map of element nodes and their coordinates were generated by custom code developed in Matlab.
The resulting meshes of the core, adhesive layer and skin are shown in Figure \ref{fig:cas_mesh}.

The mesh for the cyanoacrylate adhesive consisted of five elements, with a second-order curve at the structure boundary, as seen in Figure~\ref{fig:struct_mesh}(\textbf{c}).
This structure was connected to the skin with the non-matching interface elements with the adhesive mesh selected as a slave one.
The \ac{pzt} mesh coincided with the glue mesh and they are connected with the matching interface elements.

\begin{figure}[H]
	\begin{center}
		\includegraphics[width=0.95\textwidth]{Chapter_5/cfrp_mesh}
	\end{center}
	\caption{The meshes of \acl{hsc} components and the interfaces between them}
	\label{fig:cas_mesh}
\end{figure}
