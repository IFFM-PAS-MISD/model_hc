%% SECTION HEADER /////////////////////////////////////////////////////////////////////////////////////
\section{Mesh generation for the \acs{fcgm}}
\label{sec:honeycomb}

%% SECTION CONTENT ////////////////////////////////////////////////////////////////////////////////////
The modelled structure is composed of the following components: 2D for the core, epoxy adhesive and cyanoacrylate glue and 3D for the \ac{cfrp} plate and \acp{pzt}.
During the creation of the core mesh, special attention was taken to minimised the number of non-zero values in the matrix \(\textbf{G}\).
The core elements were selected for the slave mesh, with one spectral element dedicated to each honeycomb cell wall.
The master meshes of the skin panel and adhesive layer were divided by three rhombic elements into the area under the core cell.
The spectral element of the wall is presented in Fig.~\ref{fig:struct_mesh}(\textbf{a}))
This way, the interface nodes coincide with those on the hexagon edges (red line in Fig.~\ref{fig:struct_mesh}(\textbf{b})).
The map of element nodes and their coordinates were generated by custom code developed in Matlab.
The resulting meshes of the core, adhesive layer and skin are shown in Fig. \ref{fig:cas_mesh}.

The mesh for the cyanoacrylate adhesive consisted of five elements, with a second-order curve at the structure boundary as it is presented in Fig.~\ref{fig:struct_mesh}(\textbf{c}).
This structure was connected to the skin with the non-matching interface elements with the adhesive mesh chosen as a slave one.
The \ac{pzt} mesh coincides with the glue mesh and they are connected with the matching interface elements.
\begin{figure}[H]
	\begin{center}
		\includegraphics[width=0.95\textwidth]{Chapter_5/struct_mesh}
	\end{center}
	\caption{The mesh with the node distribution, (\textbf{a}) spectral element used for modeling the wall of the core, (\textbf{b}) excerpt of the skin plate and (\textbf{c}) cyanoacrylate glue mesh with the second-order curve at the boundary.}
	\label{fig:struct_mesh}
\end{figure}
\begin{figure}[H]
	\begin{center}
		\includegraphics[width=0.95\textwidth]{Chapter_5/cfrp_mesh}
	\end{center}
	\caption{The meshes of the \acf{hsc} components.}
	\label{fig:cas_mesh}
\end{figure}

Spatial convergence was determined after simulations for samples with elements of different order of the Legendre polynomial (see Eq. \ref{eq:nodes}).
The polynomial order for the \(\xi\times \eta\) plane in the local reference system were changed in the set: \(p-1=[4,\,5,\,7,\,9]\).
The exact order was assumed for each component to minimise the non-zero value of coupling matrix \(\textbf{G}\).
Due to the perpendicularity of the core elements to the skin, the nodes along the core thickness are fixed to five.
As a criterion for convergence, the percentage error defined as:
\begin{eqnarray}
	\delta = \frac{\sum{\left(e^{9}-e^{p-1}\right)^2}}{\sum{\left(e^{p-1}\right)^2}} \times 100\%,
	\label{eq:perc_err_conv}
\end{eqnarray}
where \(e^{9}\) is the signal envelope for the case with the elements of \(^{\mathrm{th}}\) polynomial order.
It is the most significant possible nodes number that will fit in the operating memory of the owned \ac{gpu}.
The \(e^{p-1}\) is the envelope for the observed case.
The signal envelope is obtained using
the Hilbert transform, which is defined as \cite{staszewski2004health}:
\begin{eqnarray}
	\label{eq:hilbert}
	\hat{x}(t) &=& \frac{1}{\pi}\int_{-\infty}^{+\infty}x(\tau)\frac{1}{t-\tau}\diff\tau,\\
	\label{eq:envelope}
	e(t) &=& \sqrt{x^2(t)+\hat{x}^2(t)}.
\end{eqnarray}
\nomtypeD[e]{\(e(t)\)}{Signal envelope}{-}%
\nomtypeR[t]{\(t\)}{Time vector}{-}{\unit{\second}}%
The example of the simulated signals of 100 \unit{\kHz} are presented in Fig. \ref{fig:dx_conv}\textbf{(a)}.
It can be seen that despite the good agreement of the wave speed for all cases, the amplitudes converge only for a polynomial of order 7. 
From the graph including errors \textbf{(b)}) 
Based on the graph showing simulation errors (Fig. \ref{fig:dx_conv}, the polynomial of order 8 was selected for 50 and 100 \unit{\kHz} signals, for which the error is less than 2\%.
On the other hand, the maximum possible order 9 was selected for the 150 \unit{\kHz} signal.
Tab \ref{tab:elements_nodes} contains a complete list of elements with the number of nodes on each axis \(\xi\times \eta \times \zeta\).
\begin{figure}[H]
\begin{center}
	\includegraphics[width=0.95\textwidth]{Chapter_5/dx_conv}
\end{center}
\caption{Spacial convergence for the sample, \textbf{(a)} the sensor signals of 100 \unit{\kHz} for various number of the in-plane nodes (\(n_{\xi} \times n_{\eta}\)) of the element, \textbf{(b)} percent error for the differ}
\label{fig:dx_conv}
\end{figure}
\begin{table}[H]
\small
\tabcolsep=0.5cm
\centering
\caption{\label{tab:elements_nodes}The node numbers of the sample components.}
\begin{tabular}{cccc}
	\toprule
	\multirow{3}{*}{\textbf{Component}} & \multicolumn{3}{c}{\textbf{Element nodes number}}\\
	& \multicolumn{3}{c}{\(n_{xi}\times n_{\eta} \times n_{zeta}\)}\\
	& 50 \unit{\kHz} & 100 \unit{\kHz} & 150 \unit{\kHz}\\
	\midrule
	Core & \multicolumn{2}{c}{\numproduct{8 x 5 x 1}} & \numproduct{10 x 5 x 1}\\
	Adhesive layer & \multicolumn{2}{c}{\numproduct{8 x 8 x 1}} & \numproduct{10 x 10 x 1}\\
	Skin & \multicolumn{2}{c}{\numproduct{8 x 8 x 4}} & \numproduct{10 x 10 x 1}\\
	Glue & \multicolumn{2}{c}{\numproduct{8 x 8 x 1}} & \numproduct{10 x 10 x 1}\\
	\ac{pzt} & \multicolumn{2}{c}{\numproduct{8 x 8 x 3}} & \numproduct{10 x 10 x 3}\\
	\bottomrule
	\end{tabular}
\end{table}

While the maximum length of the skin element is 6 \unit{\mm}, such a \ac{cfrp} model satisfies the condition of at least six nodes per wavelength for the \ac{a0}, as it is the shortest mode propagating in the assumed frequency range.
Table~\ref{tab:wavelength} shows the \ac{a0} wavelengths for various frequency and propagation angles.
\begin{table}[H]
	\small
	\tabcolsep=0.75cm
	%\centering
	\caption{\label{tab:wavelength}The wavelength of the \ac{a0} mode propagated in the presented \ac{cfrp} plate.}
	\begin{tabular}{cccccc}
		\toprule
		\textbf{Frequency} & \multicolumn{5}{c}{\textbf{Propagation angle}}\\
		\unit{\kHz} & \ang{0} & \ang{30} & \ang{45} & \ang{60} & \ang{90}\\
		\midrule
		50 & 16.5& 15.2&15.0&15.2&16.6\\
		100 & 10.3& 9.6&9.5&9.6&10.3\\
   	\bottomrule
		\multicolumn{6}{r}{{\scriptsize{source: Dispersion Calculator v1.9}}}
	\end{tabular}
\end{table}