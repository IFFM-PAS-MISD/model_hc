%% SECTION HEADER /////////////////////////////////////////////////////////////////////////////////////
\section{Excitation signal}
\label{sec:excitation}

%% SECTION CONTENT ////////////////////////////////////////////////////////////////////////////////////
A sine function modulated by the Hann window was chosen as the excitation signal.
It is defined as:
\begin{eqnarray}
	V_e(t) = 0.5\left(1-\cos(2\pi f_m(t-1/f_m)\right)\sin(2\pi f_ct),
\end{eqnarray}
where \(f_c\) is the carrier frequency, and \(f_m=f_c/N_c\) is the modulation frequency with \(N_c\) as the number of cycles.
\(N_c\) was assumed to be five, as a compromise between signal length in the time domain and signal width in the frequency domain.
It is because too high \(N_c\) may cause overlapping wave modes, while too low number will cause increasing signal dispersion.
Both issues can cause difficulties in signal processing for damage assessment.
The set of carrier frequencies was considered to be \(f_c=[50, 100, 150] \) \unit{\kHz}.

The convergence of the solution of the equation of motion requires time increment to be less than a critical value.
If the increment is adopted too large, the displacements immediately tend towards infinity due to increasing numerical errors.
Therefore, a maximum step value is sought for which the solution is stable. 
In the present models, it is obtained for \(\Delta t=\)\num{12.2e-3} \(\mu\)s, which represents more than 80 \unit{\MHz} of the sampling frequency.
