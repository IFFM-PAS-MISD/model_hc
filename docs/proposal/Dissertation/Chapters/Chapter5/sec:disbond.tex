%% SECTION HEADER /////////////////////////////////////////////////////////////////////////////////////
\section{GW Propagation in the Damaged Structure}
\label{sec:disbond}
%% SECTION CONTENT ////////////////////////////////////////////////////////////////////////////////////
The skin and the core disbonds were evaluated to analyse the effect of damage on GW propagation.
A rectangular region of disbonds was investigated with the length along the wave propagation varying in \(w_d=[0,10,30,50,70,100,120]\) mm, while the size in the perpendicular direction remained constant \(l_d=170\) mm.
The rectangle was centrally located between the transducers, as presented in Fig.~\ref{fig:honeycomb} (\textbf{a}).

There were two kinds of disbonds models under consideration.
The core cells were removed from the damaged area in the first model as in the mesh pictured in Fig.~\ref{fig:disbond}(\textbf{b}).
While in the second model, all components were intact, and only the interface elements between the adhesive layer and the core were decoupled within the yellow area indicated in Fig.~\ref{fig:disbond}(\textbf{c}).
\begin{figure}
	\begin{center}
		\includegraphics{Chapter_5/disbond}
	\end{center}
	\caption{The damaged area in the: (\textbf{a}) experimental sample,(\textbf{b}) numerical model with removed cells and (\textbf{c}) numerical model with interface decoupling.}
	\label{fig:disbond}
\end{figure}