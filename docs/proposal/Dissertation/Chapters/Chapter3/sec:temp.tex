%% SECTION HEADER /////////////////////////////////////////////////////////////////////////////////////
\section{Temperature Effect on the GW Propagation}
\label{sec:temp}

%% SECTION CONTENT ////////////////////////////////////////////////////////////////////////////////////

%% SUBSECTION HEADER //////////////////////////////////////////////////////////////////////////////////
The measurements of \acp{gw} are within a range of milliseconds.
The temperature changes of the inspected structure within that period are negligible.
Therefore, \ac{gw} propagation can be modelled for the stationary temperature field.
It is assumed that the temperature field can be obtained from temperature sensors at the moment of \ac{gw} excitation.
A uniform temperature field was assumed in the model for simplicity.

In order to carry out the temperature-dependent \ac{sem} simulations, only changes in the elastic modulus of the \ac{hsc} components are considered.
The density changes are neglected. For the \ac{cfrp} skin, the modulus is calculated as per the methodology described in \cite{chamis1983simplified,salamone2009guided}.
The significant changes in mechanical properties under temperature occur mainly in the polymer matrix, while the variation in the carbon fibre properties has a negligible effect on wave propagation.
In this model \cite{salamone2009guided,hopkins2012extreme}, the reduction of Young’s modulus of the resin \(E_m\) with temperature variation is assumed as:
\begin{eqnarray}
	E_m(T)=F_m E_{rm},
	\label{eq:factor_temp}
\end{eqnarray}
where \(E_{rm}\) is the Young’s modulus of resin at the reference temperature and \(F_m\) is the temperature degradation factor as proposed in \cite{chamis1983simplified}:
\begin{eqnarray}
F_m=\sqrt{\frac{T_{g0}-T}{T_{g0}-T_r}},
\label{eq:em_temp}
\end{eqnarray}
where \(T_{g0}\) is the glass transition temperature and \(T_r\) is the reference temperature.
Equation (\ref{eq:em_temp}) is also applicable to determine the elastic modulus of the adhesive layer bonding of the core to the skins, while for aluminium, the linear temperature dependence given by Hopkins et al. \cite{hopkins2012extreme} as:
\begin{eqnarray}
	E_a(T)=-\num{4e7}T+\num{8e10}.
	\label{eq:aluminium_temp}
\end{eqnarray}
The Young modulus and Poisson ratio of the sensors are in the form proposed by Lanza et al. \cite{lanza2008temperature}:
\begin{eqnarray}
	Y_{pzt}(T) & = & Y_{pzt}(T_r) + \num{16e7}(T_r-T),\\
	\nu_{pzt}(T) & = & \nu_{pzt}(T_r) + \num{13e-3}(T_r-T).
	\label{eq:pzt_temp}
\end{eqnarray}
Additionally, the piezo- and electromechanical properties are taken into account based on the temperature characteristics provided by the manufacturer.
