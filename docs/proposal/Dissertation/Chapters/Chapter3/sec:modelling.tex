%% SECTION HEADER /////////////////////////////////////////////////////////////////////////////////////
\section{Modeling of the GW Propagation in the \acp{hsc}}
\label{sec:modelling}

%% SECTION CONTENT ////////////////////////////////////////////////////////////////////////////////////

%% SUBSECTION HEADER //////////////////////////////////////////////////////////////////////////////////

In the dissertation, the \ac{hsc} composed of an aluminium honeycomb core and the skins made of \ac{cfrp} is assumed for further analysis.
The most common numerical modeling of the phenomenon of \ac{gw} in \acp{hsc} found in the literature is a calculation of the effective material properties of the honeycomb structure \cite{baid2015dispersion, mustapha2014leaky, qi2008ultrasonic,  shi1995derivation, sikdar2016guided}.
The properties are obtained from the analytical \cite{gibson1982mechanics, malek2015effective} or \ac{fem} \cite{catapano2014multi, chen2014analysis} analysis of the honeycomb \ac{rve}.
A comprehensive literature review on the homogenisation of the honeycomb structure is presented in the work of Ahmed \cite{ahmed2019homogenization}.
Replacing the core geometry with a homogeneous material has many advantages.
First and foremost, it simplifies the domain mesh so that convergence of the solution requires fewer working memory resources and increases the value of the critical time step.
In addition, the wave propagation velocity determined by the simulation is in good agreement with the experiment.

However, this method cannot adequately represent the phenomenon of propagating wave interaction in honeycomb cells.
It causes the signal energy not to dissipate as it would in a real structure.
A more precise model is the \ac{fgm} of the core. 
Ruzzene et al. presented a parametric study to evaluate the dynamic behavior of the honeycomb and cellular structures through the \ac{fem} and the application of the theory of periodic structures \cite{ruzzene2003wave}.
Recently, the simulations of the wave propagation in the \acp{hsc} have been conducted with commercially available finite element code~\cite{song2009guided, hosseini2013numerical, tian2015wavenumber, zhao2018wave}.

While the \ac{fem} based modelling of \ac{gw} requires a significant amount of memory and is time-consuming, this method becomes inefficient in the case of \ac{fgm}.
Kudela increased the computational efficiency with the model based on the time-domain \ac{sem} \cite{kudela2016parallel}.
In addition, the algorithm has been adapted for parallel computing on the \ac{gpu}, making the simulations fourteen times faster than on the \ac{cpu}.
However, this approach has two major drawbacks. One is employing solid elements with three \acp{dof} at each node to model the core walls. As a result, a $179\times160$ mm sandwich panel has over 1.5 million \acp{dof}.
Secondly, no \ac{pzt} sensors were considered in the simulation, so a concentrated force was used to generate the \ac{gw}.
To attach the transducers, the grids of the sensor and the host plate must coincide or use an interface between them. 

The disadvantages mentioned above were motivation to propose a new model of the \ac{hsc}.
In the proposed model, the core of the plate consists of \ac{2d} elements, one per cell wall.
Since the neutral plane of the elements is oriented differently concerning the global coordinate system, the local displacements vector has to be transformed accordingly.
The skin model will be developed according to the laminate theory presented by Vinson and Sierakowski \cite{vinson1993behavior}.
In addition, two interfaces are used to connect the individual \ac{hsc} components.
One with the non-matching grid was developed to join the sensors with the panel.
It was done with the novel method based on the element shape functions described in Chapter \ref{ch:sem}.
The core and skin connection was implemented with a perfect matching interface.
To the best of the author's knowledge, the presented model has not been implemented yet for the \acp{hsc}.

The parametric study conducted in the dissertation leads to the determination of a \ac{madif}, which defines the influence of the size of the composite defect on wave propagation.
In this case, the defect is assumed to be a disbond between the skin and the core.