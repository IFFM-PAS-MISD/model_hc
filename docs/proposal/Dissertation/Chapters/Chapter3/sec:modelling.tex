%% SECTION HEADER /////////////////////////////////////////////////////////////////////////////////////
\section{Modelling of the GW Propagation in the SCS}
\label{sec:modelling}

%% SECTION CONTENT ////////////////////////////////////////////////////////////////////////////////////

%% SUBSECTION HEADER //////////////////////////////////////////////////////////////////////////////////

The most common numerical modeling of the phenomenon of \ac{gw} in \acp{hsc} found in the literature is a calculation of the effective material properties of the honeycomb structure \cite{baid2015dispersion, mustapha2014leaky, qi2008ultrasonic,  shi1995derivation, sikdar2016guided}.
The properties are obtained from the analytical \cite{gibson1982mechanics, malek2015effective} or \ac{fem} \cite{catapano2014multi, chen2014analysis} analysis of the honeycomb \ac{rve}.
A comprehensive literature review on the homogenization of the honeycomb structure is presented in work of Ahmed \cite{ahmed2019homogenization}.
Replacing the core geometry with a homogeneous material has many advantages.
First and foremost, it simplifies the domain mesh so that convergence of the solution requires fewer working memory resources and increases the value of the critical time step.
In addition, the wave propagation velocity is determined by the simulation is in good agreement with the experiment.

However, this method cannot adequately represent the phenomenon of propagating wave interaction in honeycomb cells.
It causes the signal energy not to dissipate as it would in a real structure.
Song et al. \cite{song2009guided} analyzed the degree of energy dispersion function of core geometry and signal frequency.
A more accurate model will be achieved if the real geometry of the hexagonal cell is retained.
Ruzzenne et al. presented a parametric study to evaluate the dynamic behavior of the honeycomb and cellular structures through the \ac{fem} and the application of the theory of periodic structures \cite{ruzzene2003wave}.
Recently, the simulations of the wave propagation in the \acp{hsc} have been conducted with commercially available finite element code~\cite{ hosseini2013numerical,song2009guided, tian2015wavenumber, zhao2018wave}.

However, the finite element method \ac{pzt} modeling of \ac{gw} is inefficient as it requires a significant amount of memory and is time-consuming.
The computational efficiency of the \ac{fem} in case of \ac{gw} modeling in the \acp{hsc} can be improved by using the time-domain \ac{sem}.
The \ac{sem} was originally used for the numerical solution of the fluid flow in a channel by Patera \cite{patera1984spectral} but has also been successfully developed for elastic wave propagation~\cite{ostachowicz2011guided}.

Kudela proposed a model of the \ac{gw} in \acp{hsc} by the parallel implementation of the \ac{sem} \cite{kudela2016parallel}.
The wave excitation was realized by an external force applied at the point of the panel.
However, this model had a large number (1.5 million) of \acp{dof}, because cells of the core and skin plate were modeled by the \ac{3d} spectral elements; however, the simulation was limited to only one skin plate and a small dimension of the \ac{hsc} (\(179 \times 159 \) mm).

The above-mentioned drawbacks were motivation to propose a new model of the \ac{hsc}.
In the present paper, the skin plates, adhesive layers and each wall of the hexagonal core were modeled by the \ac{2d} spectral elements.
However, \ac{2d} elements have nodes only in a mid-plane; therefore, there is no direct linking between the two adjacent structures.
This connection was implemented by interface elements based on Lagrange multipliers \cite{ashwin2014formulation, fiborek20192d}.

Additionally, the signal was generated and recorded with \acp{pzt}.
A non-matching interface between the transducers and the skin was used to avoid a too complex mesh-likewise to the interfaces developed for the \ac{fem} \cite{flemisch2000elasto, flemisch2012non}. 
To the best of the authors’ knowledge, the present model has not been implemented yet for \acp{hsc}.

The parametric study conducted in the paper leads to the determination of a \ac{madif}, which defines the influence of the size of the composite defect on wave propagation.
In this case, the defect is assumed to be a disbond between the skin and the core.
