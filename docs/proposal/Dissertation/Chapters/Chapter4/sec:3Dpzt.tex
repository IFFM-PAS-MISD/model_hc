%% SECTION HEADER /////////////////////////////////////////////////////////////////////////////////////
\section{3D Model of the PZT transducers}
\label{sec:3Dpzt}

%% SECTION CONTENT ////////////////////////////////////////////////////////////////////////////////////


The displacement vector of the PZT transducer is composed of three translational displacements and  is defined as:
\begin{eqnarray}
	\left \{ \begin{array}{c}
		\textbf{u}^e(\xi,\eta,\zeta) \\
		\textbf{v}^e(\xi,\eta,\zeta) \\
		\textbf{w}^e(\xi,\eta,\zeta)
	\end{array} \right\}
	= \textbf{N}^e(\xi,\eta, \zeta)\widehat{\textbf{d}}^e
	= \sum_{l=1}^r\sum_{n=1}^q\sum_{m=1}^p\textbf{N}_m^e(\xi)\textbf{N}_n^e(\eta)\textbf{N}_l^e(\zeta)
	\left \{ \begin{array}{c}
		\widehat{\textbf{u}}^e(\xi_m,\eta_n,\zeta_l) \\
		\widehat{\textbf{v}}^e(\xi_m,\eta_n,\zeta_l) \\
		\widehat{\textbf{w}}^e(\xi_m,\eta_n,\zeta_l)
	\end{array} \right\},
	\label{eq:3D_displ}
\end{eqnarray}
where \(\widehat{\textbf{u}}^e\), \(\widehat{\textbf{v}}^e\) and 
\(\widehat{\textbf{w}}^e\) are displacements of the element nodes in \(\xi,\eta\) and \(\zeta\) direction.

The nodal strain--displacement relations are given as \cite{kudela20093d}:
\begin{eqnarray}
	\boldsymbol{\epsilon}^e=\textbf{B}_{d}^e\widehat{\textbf{d}}^e=
	\left [
	\begin{array}{ccc}
		\frac{\partial N^e}{\partial x} & 0 & 0\\
		0 & \frac{\partial N^e}{\partial y} & 0\\
		0 & 0 & \frac{\partial N^e}{\partial z}\\
		0 & \frac{\partial N^e}{\partial z} & \frac{\partial N^e}{\partial y}\\
		\frac{\partial N^e}{\partial z} & 0 & \frac{\partial N^e}{\partial x}\\
		\frac{\partial N^e}{\partial y} & \frac{\partial N^e}{\partial x} & 0
	\end{array} \right]
	\left \{ \begin{array}{c}
		\widehat{\textbf{u}}^e \\
		\widehat{\textbf{v}}^e \\
		\widehat{\textbf{w}}^e
	\end{array} \right\}.
\end{eqnarray}

The electromechanical coupling is governed by the linear constitutive equation of piezoelectric material according to~\cite{giurgiutiumicromechatronics, rekatsinas2017cubic}, and this is defined as:
\begin{eqnarray}
	\left [ 
	\begin {array}{c}
	\boldsymbol{\sigma}\\
	\textbf{D}
\end{array}\right ]=
\left [ 
\begin{array}{cc}
	\textbf{c}^E & -\textbf{e}^T \\
	\textbf{e} & \epsilon^S 
\end{array} \right ]
\left[ 
\begin{array}{c}
	\textbf{S}\\
	\textbf{E} 
\end{array} \right ],
\end{eqnarray}
where \(\boldsymbol{\sigma}\) and \(\textbf{S}\) are the stress and strain components, respectively, \(\textbf{c}^E\) is the stiffness coefficient matrix measured at zero electric field, \textbf{e} is the piezoelectric coupling tensor,  \(\boldsymbol{\epsilon}^S\) is the electric permittivity, and \textbf{E} and \textbf{D} are the electric field and electric displacement measured at zero strain.
The superscript T denotes a transpose matrix.
The electric field is defined as:
\begin{eqnarray}
\textbf{E}^e=-\textbf{B}_\phi^e \widehat{\boldsymbol{\phi}}^e = \left[ \begin{array}{c}
	\frac{\partial N^e}{\partial \xi}\\
	\frac{\partial N^e}{\partial \eta}\\
	\frac{\partial N^e}{\partial \zeta}
\end{array} \right] \widehat{\boldsymbol{\phi}}^e.
\end{eqnarray}
where \(\widehat{\boldsymbol{\phi}}^e\) is a nodal voltage of the transducer.