%% SECTION HEADER /////////////////////////////////////////////////////////////////////////////////////
\section{\Acs{pzt} modelling}
\label{sec:PZTmodel}

%% SECTION CONTENT ////////////////////////////////////////////////////////////////////////////////////
The electromechanical coupling is governed by the linear constitutive equation of piezoelectric material according to~\cite{giurgiutiu2009micromechatronics, rekatsinas2017cubic}, and this is defined as:
\begin{eqnarray}
	\left [ 
	\begin {array}{c}
	\boldsymbol{\sigma}\\
	\widetilde{\textbf{D}}
\end{array}\right ]=
\left [ 
\begin{array}{cc}
	\textbf{c}^E & -\textbf{e}^{\mathrm{T}} \\
	\textbf{e} & \epsilon^S 
\end{array} \right ]
\left[ 
\begin{array}{c}
	\boldsymbol{\varepsilon}\\
	\widetilde{\textbf{E}} 
\end{array} \right ],
\label{eq:elecmechcoupling}
\end{eqnarray}
\nomtypeR[Ep]{$\widetilde{\textbf{E}} $}{Electric field}{\(-\textbf{B}\,\boldsymbol{\phi}\)}{\unit[per-mode=symbol]{\newton\per\coulomb}}%
\nomtypeR[Dp]{$\widetilde{\textbf{D}} $}{Electric charge density displacement}{-}{\unit[per-mode=symbol]{\coulomb\per\square\metre}}%
\nomtypeG[phip]{$\boldsymbol{\phi}$}{Electric potential vector}{-}{\unit{\volt}}%
\nomtypeG[sigma]{$\boldsymbol{\sigma}$}{Stress tensor}{\(\textbf{c}\,\boldsymbol{\varepsilon}\)}{\unit{\pascal}}%
where \(\boldsymbol{\sigma}\) is the stress components, \(\textbf{c}^E\) is the stiffness coefficient matrix measured at zero electric field, \textbf{e} is the piezoelectric coupling tensor, \(\boldsymbol{\epsilon}^S\) is the electric permittivity measured at zero strain, and \(\widetilde{\textbf{E}}\) and \(\widetilde{\textbf{D}}\) are the electric field and electric charge density displacement.
The electric field of the element is defined as:
\begin{eqnarray}
\widetilde{\textbf{E}}^e=-\textbf{B}_\phi^e \widehat{\boldsymbol{\phi}}^e = \left[ \begin{array}{c}
	\frac{\partial N^e}{\partial \xi}\\
	\frac{\partial N^e}{\partial \eta}\\
	\frac{\partial N^e}{\partial \zeta}
\end{array} \right] \widehat{\boldsymbol{\phi}}^e,
\end{eqnarray}
where \(\widehat{\boldsymbol{\phi}}^e\) is a nodal voltage of the transducer. The \ac{sem} formulation of the governing equation (\ref{eq:elecmechcoupling}) is defined as:
\begin{eqnarray}
	\left [\begin{array}{cc}
		\textbf{M}_{dd} & \textbf{0}\\
		\textbf{0} & \textbf{0}
	\end{array}\right]
	\left \{\begin{array}{c}
		\widehat{\ddot{\textbf{d}}} \\
		\textbf{0}
	\end{array}\right \} +
	\left [\begin{array}{cc}
		\textbf{K}_{dd} & \textbf{K}_{d \phi}\\
		\textbf{K}_{d \phi}^{\mathrm{T}} & \textbf{K}_{\phi \phi}
	\end{array}\right]
	\left \{\begin{array}{c}
		\widehat{\textbf{d}} \\
		\widehat{\boldsymbol{\phi}}
	\end{array}\right \}  = 
	\left \{\begin{array}{c}
		\textbf{0}\\
		\widehat{\textbf{Q}}
	\end{array}\right \},
	\label{eq:pzt_sem}
\end{eqnarray}
\nomtypeR[Q]{$\textbf{Q}$}{Charge vector}{-}{\unit{\coulomb}}%
where \(\widehat{\textbf{Q}}\) is the nodal charge vector.
The mass and stiffness matrices are defined according to \ac{3d} model from section \ref{sec:3Dmodel}.
The piezoelectric coupling matrix \(\textbf{K}_{\phi \phi}^e\) and the dielectric permittivity matrix \(\textbf{K}_{d \phi}^e\) are defined as:
\begin{eqnarray}
	\textbf{K}_{d\phi}^e & = & \int_{V_e}{\textbf{B}_d^e}^{\mathrm{T}}\textbf{e}^{\mathrm{T}} \textbf{B}_{\phi}^e \diff V_e,\\
	\textbf{K}_{\phi \phi}^e & = & -\int_{V_e}{\textbf{B}_{\phi}^e}^{\mathrm{T}} 
	{\textbf{\(\epsilon\)}^S}^{\mathrm{T}} \textbf{B}_{\phi}^e \diff V_e.
\end{eqnarray}

If a vector \(\textbf{b}\) contains list of consecutive boundary nodes (corresponding to electrodes) and a vector \(\textbf{a}\) contains lists of consecutive active nodes (remaining nodes) of the \ac{pzt}, the electrical potential vector and the charge vector can be rewritten as:
\begin{eqnarray}
	\widehat{\boldsymbol{\phi}} & = & \left \{\begin{array}{cc}
		\widehat{\boldsymbol{\phi}}(\textbf{b}) &
		\widehat{\boldsymbol{\phi}}(\textbf{a})
	\end{array}\right \}^{\mathrm{T}},\\
	\widehat{\textbf{Q}} & = & \left \{\begin{array}{cc}
		\widehat{\textbf{Q}}(\textbf{b}) & \textbf{0}
	\end{array}\right \}^{\mathrm{T}}.
	\label{eq:phi_Q}
\end{eqnarray}
Then, piezoelectric part of Eq.~(\ref{eq:pzt_sem}) is expressed as:
\begin{eqnarray}
	\begin{split}
		\left [\begin{array}{cc}
			\textbf{K}_{d \phi}(:,\textbf{b}) &
			\textbf{K}_{d \phi}(:,\textbf{a})
		\end{array}\right]^{\mathrm{T}}
		\widehat{\textbf{d}} & +
		\left [\begin{array}{cc}
			\textbf{K}_{\phi \phi}(\textbf{b},\textbf{b}) & \textbf{K}_{\phi 		\phi}(\textbf{b},\textbf{a})\\
			\textbf{K}_{\phi \phi}(\textbf{a},\textbf{b}) & \textbf{K}_{\phi \phi}(\textbf{a},\textbf{a})
		\end{array}\right]
		\left \{\begin{array}{c}
			\widehat{\boldsymbol{\phi}}(\textbf{b}) \\
			\widehat{\boldsymbol{\phi}}(\textbf{a})
		\end{array}\right \}\\ 
		& = \left \{\begin{array}{c}
			\widehat{\textbf{Q}}(\textbf{b}) \\
			\textbf{0}
		\end{array}\right \},
	\end{split}
	\label{eq:pztboundary}
\end{eqnarray}
where the notation \(\textbf{K}(\textbf{r},\textbf{c})\) uses vectors \(\textbf{r}\) and \(\textbf{c}\) to extract rows and columns from the matrix \(\textbf{K}\), respectively, and \((:)\) means all rows or columns of \(\textbf{K}\).
The electrical potential of the free nodes can be extracted from Eq.~(\ref{eq:pztboundary}):
\begin{eqnarray}
	\widehat{\boldsymbol{\phi}}(\textbf{a}) = -\textbf{K}_{\phi\phi}^{-1}(\textbf{a},\textbf{a})\left[\textbf{K}_{\phi d}(\textbf{a},:) \widehat{\textbf{d}} + \textbf{K}_{\phi\phi}(\textbf{a},\textbf{b})\widehat{\boldsymbol{\phi}}(\textbf{b}) \right].
	\label{eq:freePotetial}
\end{eqnarray}
If the \ac{pzt} acts as an actuator, the electrical potential of the electrode nodes has the values of the applied signal.
As one of the electrodes is grounded, the potential is zero.
Therefore, the potential vector can be written as:
\begin{eqnarray}
	\widehat{\boldsymbol{\phi}}(\textbf{b}) = \left \{\begin{array}{cc}
		\widehat{\boldsymbol{\phi}}(\textbf{b}_v) &
		\widehat{\boldsymbol{\phi}}(\textbf{b}_g)
	\end{array}\right \}^{\mathrm{T}}=\left \{\begin{array}{cc}
	\textbf{V}(t) & \textbf{0}
	\end{array}\right \}^{\mathrm{T}},
	\label{eq:phi_V}
\end{eqnarray}
where \(\textbf{b}_v\) is a list of nodes of the applied electrode and \(\textbf{b}_g\) is a list of nodes of the grounded electrode.
Substituting Eq. (\ref{eq:phi_V}) into Eq. (\ref{eq:freePotetial}) and Eq. (\ref{eq:pztboundary}), induced stiffness of the actuator is obtained:
\begin{eqnarray}
	\textbf{K}_{a}=-\textbf{K}_{d\phi}(:,\textbf{a})\,\textbf{K}_{\phi \phi}^{-1}(\textbf{a},\textbf{a})\,\textbf{K}_{\phi \phi} (\textbf{a},\textbf{b}).
\end{eqnarray}
Hence, the equivalent mechanical force vector of the applied voltage of the piezoelectric actuator equals:
\begin{eqnarray}
	\widehat{\textbf{f}}_{a}=-\textbf{K}_{a}\,\widehat{\boldsymbol{\phi}}(\textbf{b}).
	\label{eq:f_act}
\end{eqnarray}
\nomtypeR[force_act]{$\textbf{f}_{a}$}{Actuator equivalent force vector}{\(-\textbf{K}_a\,\boldsymbol{\phi}(b)\)}{\unit{\newton}}%
In the case of the open circuit sensor one electrode is grounded so the electric potential of the free nodes becomes as:
\begin{eqnarray}
	\widehat{\boldsymbol{\phi}}(\textbf{a}) = -\textbf{K}_{\phi\phi}^{-1}(\textbf{a},\textbf{a})\,\textbf{K}_{\phi d}(\textbf{a},:)\,\widehat{\textbf{d}}.
	\label{eq:sensorPotetial}
\end{eqnarray}
The induced stiffness of the sensor can be written as:
\begin{eqnarray} \textbf{K}_s=\textbf{K}_{d \phi}(:,\textbf{a})\,\textbf{K}_{\phi \phi}^{-1} (\textbf{a},\textbf{a})\,\textbf{K}_{\phi d}(\textbf{a},:).
\end{eqnarray}
To obtain the sensor response, the nodal electric potential must be integrated over the electrode surface as follow:
\begin{eqnarray}
	\boldsymbol{\phi}(t) = \int_{\Omega_s}\widehat{\phi} \diff\Omega_s.
	\label{eq:sensorResponse}
\end{eqnarray}