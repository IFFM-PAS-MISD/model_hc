%% SECTION HEADER /////////////////////////////////////////////////////////////////////////////////////
\section{Time Integration}
\label{sec:time}

%% SECTION CONTENT ////////////////////////////////////////////////////////////////////////////////////
Assuming \(\textbf{b}\) and \(\textbf{f}\) represent order lists of the electrode nodes and free nodes of the \ac{pzt}, respectively, the electrical potential vector is rewritten:
\begin{eqnarray}
	\widehat{\boldsymbol{\phi}} = \left \{\begin{array}{cc}
		\widehat{\boldsymbol{\phi}}(\textbf{b}) &
		\widehat{\boldsymbol{\phi}}(\textbf{f})
	\end{array}\right \}^T.
	\label{eq:potential}
\end{eqnarray}

Then, Equation~(\ref{eq:piezocoupling}) is expressed as:
\begin{eqnarray}
	\left [\begin{array}{c}
		\textbf{K}_{\phi d}(\textbf{b},:) \\
		\textbf{K}_{\phi d}(\textbf{f},:)
	\end{array}\right]
	\left \{\widehat{\textbf{d}}\right\} +
	\left [\begin{array}{cc}
		\textbf{K}_{\phi \phi}(\textbf{b},\textbf{b}) & \textbf{K}_{\phi \phi}(\textbf{b},\textbf{f})\\
		\textbf{K}_{\phi \phi}(\textbf{f},\textbf{b}) & \textbf{K}_{\phi \phi}(\textbf{f},\textbf{f})
	\end{array}\right]
	\left \{\begin{array}{c}
		\widehat{\boldsymbol{\phi}}(\textbf{b}) \\
		\widehat{\boldsymbol{\phi}}(\textbf{f})
	\end{array}\right \} = 
	\left \{\begin{array}{c}
		\textbf{Q} \\
		\textbf{0}
	\end{array}\right \},
	\label{eq:piezoboundary}
\end{eqnarray} 
where the notation \(\textbf{K}(\textbf{r},\textbf{c})\) uses vectors \(\textbf{r}\) and \(\textbf{c}\) to extract rows and columns from the matrix \(\textbf{K}\), respectively, and \((:)\) means all rows or columns of \(\textbf{K}\).
The electrical potential of the free nodes can be extracted from Equation~(\ref{eq:piezoboundary}):
\begin{eqnarray}
	\widehat{\boldsymbol{\phi}}(\textbf{f}) = -\textbf{K}_{\phi\phi}^{-1}(\textbf{f},\textbf{f})\left[\textbf{K}_{\phi d}(\textbf{f},:) \widehat{\textbf{d}} + \textbf{K}_{\phi\phi}(\textbf{f},\textbf{b})\widehat{\boldsymbol{\phi}}(\textbf{b}) \right].
	\label{eq:freePotetial}
\end{eqnarray}

Substituting Equations~(\ref{eq:potential}) and (\ref{eq:freePotetial}) into Equation~(\ref{eq:motion}), the equation of motion can be rearranged into the form:
\begin{eqnarray}
	\textbf{M}_{dd} \widehat{\ddot{\textbf{d}}} + \textbf{C}_{dd} \widehat{\dot{\textbf{d}}} + (\textbf{K}_{dd}-\textbf{K}_{s}) \widehat{\textbf{d}}  = \textbf{F} + \textbf{K}_{a} \widehat{\boldsymbol{\phi}}(b) - \textbf{G}^T \boldsymbol{\lambda},
	\label{eq:motionD}
\end{eqnarray}
where  \(\textbf{K}_a=\textbf{K}_{d\phi}(:,f)\textbf{K}_{\phi \phi}^{-1}(f,f)\textbf{K}_{\phi \phi}(\textbf{f},\textbf{b})-\textbf{K}_{d\phi}(:,\textbf{b})\), \(\textbf{K}_s=\textbf{K}_{d \phi}(:,\textbf{f})\textbf{K}_{\phi \phi}^{-1}(\textbf{f},\textbf{f})\textbf{K}_{\phi d}(\textbf{f},:)\).
The unknown displacement vector \(\widehat{\textbf{d}}_t\) is found using a central difference algorithm \cite{kudela20093d}.
Thus, Equation~(\ref{eq:motionD}) is rewritten as:
\begin{equation}
	\begin{array}{c}
		\left(\frac{1}{\Delta t^2}\textbf{M}_{dd}+\frac{1}{2\Delta t}\textbf{C}_{dd} \right)\widehat{\textbf{d}}_{t+\Delta t}=
		\textbf{F}_t+\textbf{K}_a\widehat{\boldsymbol{\phi}}_t(b)-\left( \textbf{K}_{dd}-\textbf{K}_s\right)\widehat{\textbf{d}}_t+\\
		+\frac{2}{\Delta t^2}\textbf{M}_{dd}\widehat{\textbf{d}}_t-\left(\frac{1}{\Delta t^2}\textbf{M}_{dd}-\frac{1}{2\Delta t}\textbf{C}_{dd}\right)\widehat{\textbf{d}}_{t-\Delta t}-\textbf{G}^T\boldsymbol{\lambda}_t,
	\end{array}
	\label{eq:CDE}
\end{equation}
where \(\Delta t\) is the time increment.

Imposing the constrain Equation~(\ref{eq:cond_disp}), the vector of Lagrange multipliers \(\boldsymbol{\lambda}_t\) can be extracted from Equation~(\ref{eq:CDE}): 
\begin{eqnarray}
	\boldsymbol{\lambda}_t = {\left(\textbf{G}\textbf{L}_+^{-1}\textbf{G}^T \right)}^{-1}\textbf{G}\textbf{L}_+^{-1} \Bigg[ \textbf{F}_t+\textbf{K}_a\widehat{\boldsymbol{\phi}}_t(b)+\left.\left(\frac{2}{\Delta t^2}\textbf{M}_{dd}-\textbf{K}_{dd}+\textbf{K}_s\right)\widehat{\textbf{d}}_t -\textbf{L}_-\widehat{\textbf{d}}_{t-\Delta t} \right],
	\label{eq:lambda}
\end{eqnarray}
where \(\textbf{L}_{\pm}=\frac{1}{\Delta t^2}\textbf{M}_{dd}\pm\frac{1}{2\Delta t}\textbf{C}_{dd}\).