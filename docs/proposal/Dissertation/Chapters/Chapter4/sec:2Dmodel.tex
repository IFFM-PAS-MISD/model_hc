%% SECTION HEADER /////////////////////////////////////////////////////////////////////////////////////
\section{2D Spectral Modelling}
\label{sec:2Dmodel}

%% SECTION CONTENT ////////////////////////////////////////////////////////////////////////////////////

According to the first-order shear deformation theory~\cite{reissner1945effect, mindlin1951influence}, the displacement field is expressed as:
\begin{eqnarray}
	\left \{ \begin{array}{c}
		\textbf{u}^e(\xi,\eta) \\
		\textbf{v}^e(\xi,\eta) \\
		\textbf{w}^e(\xi,\eta)
	\end{array} \right\} = 
	\left \{ \begin{array}{c}
		\textbf{u}_0^e(\xi,\eta) + z\boldsymbol{\varphi}_x^e(\xi,\eta)\\
		\textbf{v}_0^e(\xi,\eta) + z\boldsymbol{\varphi}_y^e(\xi,\eta)\\
		\textbf{w}_0^e(\xi,\eta) \\
	\end{array} \right\},
\end{eqnarray}
where \(\textbf{u}_0^e\), \(\textbf{v}_0^e\) and \(\textbf{w}_0^e\) are nodal displacements, \(\boldsymbol{\varphi}_x^e\), \(\boldsymbol{\varphi}_y^e\) are the rotations of the normal to the mid-plane with respect to the axes \textit{x} and \textit{y}, respectively. They are defined as:
\begin{eqnarray}
	\left \{\begin{array}{c}
		\textbf{u}_0^e(\xi,\eta) \\
		\textbf{v}_0^e(\xi,\eta) \\
		\textbf{w}_0^e(\xi,\eta) \\
		\boldsymbol{\varphi}_x^e(\xi,\eta) \\
		\boldsymbol{\varphi}_y^e(\xi,\eta)
	\end{array} \right\}
	= \textbf{N}^e(\xi,\eta)\widehat{\textbf{d}}^e
	= \sum_{n=1}^q\sum_{m=1}^p\textbf{N}_m^e(\xi)\textbf{N}_n^e(\eta)
	\left \{ \begin{array}{c}
		\widehat{\textbf{u}}_0^e \\
		\widehat{\textbf{v}}_0^e \\
		\widehat{\textbf{w}}_0^e \\
		\widehat{\boldsymbol{\varphi}}_x^e \\
		\widehat{\boldsymbol{\varphi}}_y^e
	\end{array} \right \},
\end{eqnarray}
where $\widehat{\textbf{d}}^e$ is a nodal displacements vector of the element $e$.

The nodal bending strain--displacement relations are given in the form:
\begin{eqnarray}
	\boldsymbol{\epsilon}_b^e =
	\textbf{B}_b^e\widehat{\textbf{d}}^e = 
	\left [
	\begin{array}{ccccc}
		\frac{\partial N^e}{\partial x} & 0 & 0 & 0 & 0\\
		0 & \frac{\partial N^e}{\partial y} & 0 & 0 & 0\\
		\frac{\partial N^e}{\partial y} & \frac{\partial N^e}{\partial x} & 0 & 0 & 0\\
		0 & 0 & 0 & -\frac{\partial N^e}{\partial x} & 0\\
		0 & 0 & 0 & 0 & -\frac{\partial N^e}{\partial y}\\
		0 & 0 & 0 & -\frac{\partial N^e}{\partial y} & -\frac{\partial N^e}{\partial x}
	\end{array} \right]
	\left \{ \begin{array}{c}
		\widehat{\textbf{u}}_0^e \\
		\widehat{\textbf{v}}_0^e \\
		\widehat{\textbf{w}}_0^e \\
		\widehat{\boldsymbol{\varphi}}_x^e \\
		\widehat{\boldsymbol{\varphi}}_y^e
	\end{array} \right\}.
\end{eqnarray}

The nodal shear strain--displacement relations are given in the form:
\begin{eqnarray}
	\boldsymbol{\epsilon}_s^e =
	\textbf{B}_s^e\widehat{\textbf{d}}^e = 
	\left [
	\begin{array}{ccccc}
		0 & 0 & \frac{\partial N^e}{\partial y} & -1 & 0\\
		0 & 0 & \frac{\partial N^e}{\partial y} & 0 & -1
	\end{array} \right]
	\left \{ \begin{array}{c}
		\widehat{\textbf{u}}_0^e \\
		\widehat{\textbf{v}}_0^e \\
		\widehat{\textbf{w}}_0^e \\
		\boldsymbol{\varphi}_x^e \\
		\boldsymbol{\varphi}_y^e
	\end{array} \right\}.
\end{eqnarray}

The mass and stiffness matrices for \ac{2d} elements are defined as:
\begin{eqnarray}
	\textbf{M}_{dd}^e & = &
	\left [
	\begin{array}{cc}
		\textbf{M}^e & 0\\
		0 & \textbf{J}^e
	\end{array}
	\right] =
	\int_{\Omega_e}\textbf{N}^T\rho
	\left [
	\begin{array}{ccccc}
		h & 0 & 0 & 0 & 0 \\
		& h & 0 & 0 & 0 \\
		&  & h & 0 & 0\\
		&  &  & \frac{h^3}{12} & 0\\
		Sym. &  &  &  & \frac{h^3}{12}
	\end{array} \right]
	\textbf{N} \diff\Omega_e,\\
	\textbf{K}_{dd}^e & = & \int_{\Omega_e}{\textbf{B}_b^e}^T
	\left[
	\begin{array}{cc}
		\textbf{A} & \textbf{B}\\
		\textbf{B} & \textbf{D}
	\end{array} \right]
	\textbf{B}_b^e \diff \Omega_e+\int_{\Omega_e}{\textbf{B}_s^e}^T\hat{\textbf{A}}\textbf{B}_s^e\diff \Omega_e,
\end{eqnarray}
where \(h=h_t+h_b\) is the element thickness, while \(h_{t(b)}\) is the distance between mid-plane and top (bottom) surface of the element, and \(\Omega_e\) is the element area:
\begin{eqnarray}
	\textbf{A} & = & \textbf{c}_{ij}\,(h_t-h_b),\qquad i,j=1,2,6\nonumber,\\
	\textbf{B} & = & 1/2\, \textbf{c}_{ij}\,(h_t^2-h_b^2),\qquad i,j=1,2,6\nonumber,\\
	\textbf{D} & = & 1/3\, \textbf{c}_{ij}\,(h_t^3-h_b^3),\qquad i,j=1,2,6\nonumber,\\
	\hat{\textbf{A}} & = & 5/4\, \textbf{c}_{ij}\,\left[h_t-h_b-4/3\left(h_t^3-h_b^3\right)/h^2\right],\qquad i,j=4,5.
\end{eqnarray}
