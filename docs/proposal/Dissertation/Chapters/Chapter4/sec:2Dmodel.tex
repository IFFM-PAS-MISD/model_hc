%% SECTION HEADER /////////////////////////////////////////////////////////////////////////////////////
\section{\Acl{2d} spectral element}
\label{sec:2Dmodel}

%% SECTION CONTENT ////////////////////////////////////////////////////////////////////////////////////
The shell element has five \acp{dof} at every node, three translations and two rotations as it is shown in Figure~\ref{fig:2d_se}.
All the nodes are located in the neutral surface of the element.
\begin{figure}
	\begin{center}
		\includegraphics[width=0.8\textwidth]{Chapter_4/2d_se}
	\end{center}
	\caption{\Acl{2d} spectral element}
	\label{fig:2d_se}
\end{figure}
According to the first-order shear deformation theory~\cite{reissner1945effect, mindlin1951influence}, the displacement field is expressed as
\begin{eqnarray}
	\left \{ \begin{array}{c}
		\textbf{u}^e\\
		\textbf{v}^e\\
		\textbf{w}^e
	\end{array} \right\} = 
	\left \{ \begin{array}{c}
		\textbf{u}_0^e - z\boldsymbol{\varphi}_x^e\\
		\textbf{v}_0^e - z\boldsymbol{\varphi}_y^e\\
		\textbf{w}_0^e\\
	\end{array} \right\},
\nomtypeR[uvw]{$u,\,v,\,w$}{Displacement in x-, y- and z-axis direction}{}{\unit{\metre}}%
\nomtypeG[phi]{$\varphi^e$}{Element rotation of the normal to the mid-plane}{}{\unit{\radian}}%
\end{eqnarray}
where \(\textbf{u}_0^e\), \(\textbf{v}_0^e\) and \(\textbf{w}_0^e\) are element displacements, \(\boldsymbol{\varphi}_x^e\), \(\boldsymbol{\varphi}_y^e\) are the element rotations of the normal to the mid-plane with respect to the axes \textit{x} and \textit{y}, respectively. They are defined as
\begin{eqnarray}
	\left \{\begin{array}{c}
		\textbf{u}_0^e\\
		\textbf{v}_0^e\\
		\textbf{w}_0^e\\
		\boldsymbol{\varphi}_x^e\\
		\boldsymbol{\varphi}_y^e
	\end{array} \right\}
	= \textbf{N}^e(\xi,\eta)\widehat{\textbf{d}}^e
	= \sum_{n=1}^{q+1}\sum_{m=1}^{p+1}\textbf{N}_m^e(\xi)\textbf{N}_n^e(\eta)
	\left \{ \begin{array}{c}
		\widehat{\textbf{u}}_0^e \\
		\widehat{\textbf{v}}_0^e \\
		\widehat{\textbf{w}}_0^e \\
		\widehat{\boldsymbol{\varphi}}_x^e \\
		\widehat{\boldsymbol{\varphi}}_y^e
	\end{array} \right \},
\end{eqnarray}
where \(n\) and \(m\) are the nodes number in \(\xi\) and \(\eta\) direction, respectively, \(q\) is the Legendre polynomial degree,  $\widehat{\textbf{d}}^e$ is a nodal displacements vector of the element $e$.
The element bending strain-displacement relations are given in the form
\begin{eqnarray}
	\boldsymbol{\varepsilon}_b^e = 
	\left \{ \begin{array}{c}
		\frac{\partial \widehat{\textbf{u}}_0^e}{\partial x} \\
		\frac{\partial \widehat{\textbf{v}}_0^e}{\partial y} \\
		\frac{\partial \widehat{\textbf{u}}_0^e}{\partial y} - \frac{\partial \widehat{\textbf{v}}_0^e}{\partial x}\\
		\frac{\partial \widehat{\boldsymbol{\varphi}}_x^e}{\partial x} \\
		\frac{\partial \widehat{\boldsymbol{\varphi}}_y^e}{\partial y} \\
		\frac{\partial \widehat{\boldsymbol{\varphi}}_x^e}{\partial y} - \frac{\partial \widehat{\boldsymbol{\varphi}}_y^e}{\partial x}
	\end{array} \right\} = 
	\textbf{B}_b^e\widehat{\textbf{d}}^e = 
	\left [
	\begin{array}{ccccc}
		\frac{\partial N^e}{\partial x} & 0 & 0 & 0 & 0\\
		0 & \frac{\partial N^e}{\partial y} & 0 & 0 & 0\\
		\frac{\partial N^e}{\partial y} & \frac{\partial N^e}{\partial x} & 0 & 0 & 0\\
		0 & 0 & 0 & -\frac{\partial N^e}{\partial x} & 0\\
		0 & 0 & 0 & 0 & -\frac{\partial N^e}{\partial y}\\
		0 & 0 & 0 & -\frac{\partial N^e}{\partial y} & -\frac{\partial N^e}{\partial x}
	\end{array} \right]
	\left \{ \begin{array}{c}
		\widehat{\textbf{u}}_0^e \\
		\widehat{\textbf{v}}_0^e \\
		\widehat{\textbf{w}}_0^e \\
		\widehat{\boldsymbol{\varphi}}_x^e \\
		\widehat{\boldsymbol{\varphi}}_y^e
	\end{array} \right\}.
\end{eqnarray}
The element shear strain--displacement relations are given in the form
\begin{eqnarray}
	\boldsymbol{\varepsilon}_s^e = 
	\left \{ \begin{array}{c}
		\frac{\partial \widehat{\textbf{w}}_0^e}{\partial x} - \widehat{\boldsymbol{\varphi}}_x^e\\
		\frac{\partial \widehat{\textbf{w}}_0^e}{\partial y} - \widehat{\boldsymbol{\varphi}}_y^e
	\end{array} \right\} = 
	\textbf{B}_s^e\widehat{\textbf{d}}^e = 
	\left [
	\begin{array}{ccccc}
		0 & 0 & \frac{\partial N^e}{\partial y} & -N^e & 0\\
		0 & 0 & \frac{\partial N^e}{\partial y} & 0 & -N^e
	\end{array} \right]
	\left \{ \begin{array}{c}
		\widehat{\textbf{u}}_0^e \\
		\widehat{\textbf{v}}_0^e \\
		\widehat{\textbf{w}}_0^e \\
		\widehat{\boldsymbol{\varphi}}_x^e \\
		\widehat{\boldsymbol{\varphi}}_y^e
	\end{array} \right\}.
\nomtypeD[Eps]{$\boldsymbol{\varepsilon}$}{Strain tensor}{}%
\end{eqnarray}
The mass and stiffness matrices for \ac{2d} elements are defined as:
\begin{eqnarray}
	\textbf{M}_{dd}^e & = &
	\left [
	\begin{array}{cc}
		\textbf{M}^e & 0\\
		0 & \textbf{J}^e
	\end{array}
	\right] =
	\rho\int_{\Omega^e}\textbf{N}^{\mathrm{T}}
	\left [
	\begin{array}{ccccc}
		h^e & 0 & 0 & 0 & 0 \\
		& h^e & 0 & 0 & 0 \\
		&  & h^e & 0 & 0\\
		\multicolumn{3}{c}{\mathrm{Sym.}} & \frac{{h^e}^3}{12} & 0\\
		&  &  &  & \frac{{h^e}^3}{12}
	\end{array} \right]
	\textbf{N} \diff\Omega^e,\\
	\textbf{K}_{dd}^e & = & \int_{\Omega^e}{\textbf{B}_b^e}^{\mathrm{T}}
	\left[
	\begin{array}{cc}
		\check{\textbf{A}} & \check{\textbf{B}}\\
		\check{\textbf{B}} & \check{\textbf{D}}
	\end{array} \right]
	\textbf{B}_b^e \diff \Omega^e+\int_{\Omega^e}{\textbf{B}_s^e}^{\mathrm{T}}\hat{\textbf{A}}\textbf{B}_s^e\diff \Omega^e,
	\nomtypeR[B]{$\textbf{B}$}{Strain-displacement matrix}{}{\unit{\per\meter}}%
	\nomtypeR[he]{$h^e$}{Element thickness}{}{\unit{\meter}}%
	\nomtypeG[Omega]{$\Omega^e$}{Element area}{}{\unit{\square\meter}}%
\end{eqnarray}
where \(h^e=h_t+h_b\) is the element thickness, while \(h_{t(b)}\) is the distance between mid-plane and top (bottom) surface of the element, and \(\Omega_e\) is the element area.
The superscript T denotes a transpose matrix and the sub-matrices \(\check{\textbf{A}},\,\hat{\textbf{A}},\,\check{\textbf{B}}\,\mathrm{and}\,\check{\textbf{D}}\) are defined as:
\begin{eqnarray}
	\begin{array}{cclcl}
	\check{\textbf{A}} & = & \textbf{c}_{ij}\,(h_t-h_b) & \mathrm{for} & i,j=1,2,6,\\
	\check{\textbf{B}} & = & \textbf{c}_{ij}\,(h_t^2-h_b^2)/2 & \mathrm{for} & i,j=1,2,6,\\
	\check{\textbf{D}} & = & \textbf{c}_{ij}\,(h_t^3-h_b^3)/3 & \mathrm{for} & i,j=1,2,6,\\
	\hat{\textbf{A}} & = & 5/4\,\textbf{c}_{ij}\,\left[h_t-h_b-4/3\left(h_t^3-h_b^3\right)/{h^e}^2\right] & \mathrm{for} & i,j=4,5.
	\end{array}
\nomtypeR[c]{$\textbf{c}$}{Elasticity tensor}{}{\unit[per-mode = symbol]{\newton\per\square\meter}}%
\end{eqnarray}