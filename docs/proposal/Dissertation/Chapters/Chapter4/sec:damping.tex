%% SECTION HEADER /////////////////////////////////////////////////////////////////////////////////////
\section{Structural damping}
\label{sec:damping}

%% SECTION CONTENT ////////////////////////////////////////////////////////////////////////////////////
Propagating waves in the structure attenuate due to many factors, including geometric spreading, material damping, and dissipation into the adjacent domain.
This study adopted the Rayleigh damping model for the \ac{cfrp} skin and adhesive layer, while the damping for the aluminium core and the \ac{pzt} was neglected.
According to \cite{wandowski2017guided}, the Rayleigh damping model is defined as
\begin{eqnarray}
	\textbf{D}_{dd}^e = \alpha_M \textbf{M}_{dd}^e + \beta_K \textbf{K}_{dd}^e,
	\label{eq:damping}
	\nomtypeD[alpha]{$\alpha_M$}{Mass-proportionality damping coefficient}{}%
	\nomtypeD[beta]{$\beta_K$}{Stiffness-proportionality damping coefficient}{}%
\end{eqnarray}
where \(\alpha_M\) and \(\beta_K\) are the mass- and stiffness- proportionality coefficients.
In the presented model, \(\beta_K\) was assumed to be zero to ensure that matrix \(\textbf{D}_{dd}\) remains diagonal \cite{schulte2011simulation, wandowski2017guided}.
This assumption gives a good approximation when considering a single mode and a specific frequency. 
Ramadas showed slight differences in Lamb wave attenuation by analysing three models, i.e. mass-, stiffness-proportional and the sum of both \cite{ramadas2011modelling}.
However, due to the diagonal damping matrix, the mass-proportional model is computationally more efficient than the other models.