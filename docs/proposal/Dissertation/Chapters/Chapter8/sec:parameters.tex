%% SECTION HEADER ///////////////////////////////////////////////////////////////////////////////////
\section{The numerical simulations of the \acl{gw} under various parameters of the panel}
\label{sec:parameters}

%% SECTION CONTENT ////////////////////////////////////////////////////////////////////////////////////
In addition to the previous analyses, a parametric study was conducted to determine the effects of various component parameters on \ac{gw} propagation in \ac{hsc}. The following parameters were taken into account:
\begin{enumerate}
	\item The \ac{pzt} parameters:
	\begin{itemize}
		\item placement in relation to the core cell
		\item charge constant
		\item dielectric permittivity.
	\end{itemize}
	\item The \ac{cfrp} and the adhesive layer parameters:
		\begin{itemize}
		\item \ac{cfrp} fibre volume fraction
		\item adhesive layer thickness.
		\end{itemize}
	\item The core parameters:
	\begin{itemize}
		\item core height and width
		\item wall thickness
		\item core rotation angle regarding to wave propagation.
	\end{itemize}
\end{enumerate}

\subsection{The \acl{pzt} parameters}
Computer simulations were conducted to determine the impact of the \acp{pzt} placement relative to the core cell.
Seven transducer positions were considered, according to the schematic in Figure~\ref{fig:pzt_place}(\textbf{b}).
The actuator and sensor position relative to the cell was identical to maintain the same wave propagation distance for each case.

As seen in Figure~\ref{fig:pzt_place}(\textbf{a}), the highest amplitude of the \ac{s0} was obtained for the case with the \acp{pzt} placed in the middle of the cell.
The lowest amplitudes were obtained for the cases where the transducer midpoint lay on the core wall.
It is due to the higher stiffness underneath the \ac{pzt}, leading to less displacements.
In the case of the \ac{a0}, the signal amplitudes for position \#2 and \#5 are much lower than the rest cases.
The sensors position did not affect the measurement while monitoring the sample, as they were permanently attached to the structure. 
Nonetheless, the \acp{pzt} placement is worth considering to optimise the reference response.
\begin{figure}[!tbh]
	\begin{center}
		\includegraphics[width=0.95\textwidth]{Chapter_8/pzt_place}
	\end{center}
	\caption{The \acl{gw} propagation in the \acl{hsc} (\textbf{a}), sensor responses for (\textbf{b}) different \acf{pzt} placement in relation to the core cell}
	\label{fig:pzt_place}
\end{figure}

Changes in amplitude under the influence of different values of the piezoelectric charge constant and dielectric permittivity are presented in Figure~\ref{fig:pzt_d} and ~\ref{fig:pzt_eps}, respectively.
Both parameters have a significant effect on the signal amplitude both the \ac{s0} and \ac{a0}.
The values do not just change with ambient temperature but also degrade over service time \cite{barzegar2001aging, deangelis2006p2o}.
Therefore, it is advisable that a sensor self-diagnostic tool should be included in the \ac{shm} system. 
An \ac{emi}-based method could be a good solution as it was suggested for self-diagnostic of damaged transducers by Jiang et al. \cite{jiang2021electromechanical}.

\begin{figure}[!tbh]
	\begin{center}
		\includegraphics[width=0.95\textwidth]{Chapter_8/pzt_d}
	\end{center}
	\caption{The signal envelopes for \textbf{the piezoelectric charge constant} in the range of 80\%-120\% of the reference value}
	\label{fig:pzt_d}
\end{figure}

\begin{figure}[!tbh]
	\begin{center}
		\includegraphics[width=0.95\textwidth]{Chapter_8/pzt_eps}
	\end{center}
	\caption{The signal envelopes for \textbf{the piezoelectric dielectric permittivity} in the range of 80\%-120\% of the reference value}
	\label{fig:pzt_eps}
\end{figure}

\subsection{The \acl{cfrp} skin and the adhesive layer parameters}

The effect of the skin properties on wave propagation in \ac{hsc} was analysed for different carbon fibre volume fractions in the composite.
This parameter affects the mode velocity due to the change in effective modulus of elasticity and density of the plate.
A small change in modulus of elasticity leads to significant signal differences within the interference of wave reflections as observed in Figure~\ref{fig:skin_volume}.
It affected the \ac{madif} determination since the full-length signal was taken into account.
\begin{figure}[!tbh]
	\begin{center}
		\includegraphics[width=0.95\textwidth]{Chapter_8/skin_volume}
	\end{center}
		\caption{The signal envelopes for various \textbf{reinforcing fibres volume fractions} in the range 45-50\%}
	\label{fig:skin_volume}
\end{figure}
\pagebreak

Figure~\ref{fig:adhesive_thickness} depicts the effect of the adhesive layer thickness.
The greater the adhesive thickness, the slower both modes propagate.
There is also a noticeable decrease in the \ac{a0} amplitude, while the \ac{s0} amplitude barely changes.
It is because the \ac{a0} displacements are mainly out-of-plane, so more energy of the wave leaks into the adhesive, and it is attenuated in low-stiffness material.

\begin{figure}[!tbh]
	\begin{center}
		\includegraphics[width=0.95\textwidth]{Chapter_8/adhesive_thickness}
	\end{center}
	\caption{The signal envelopes for various \textbf{adhesive thicknesses} in the range 200-500 \(\mu\)m}
	\label{fig:adhesive_thickness}
\end{figure}

\subsection{The core parameters}
In the study of the core geometry influence on wave propagation in \ac{hsc}, four parameters were considered as follows:
\begin{itemize}
	\item core height \(g=[10.5,\,12.5,\,14.5,\,16.5,\,18.5]\) \unit{\mm}
	\item core width \(l_1=[5.0,\,6.0,\,7.0,\,8.0,\,9.0]\) \unit{\mm}
	\item wall thickness \(w_c=[100,\,150,\,200,\,250,\,300]\) \unit{\micro\m}
	\item core rotation angle regarding to wave propagation [\ang{0}, \ang{15}, \ang{30}, \ang{45}, \ang{60}, \ang{75}, \ang{90}].
\end{itemize}

The signal envelopes shown in Figures~\ref{fig:core_height}, \ref{fig:core_size}, \ref{fig:core_thickness} and \ref{fig:core_rotation} illustrate that the analysed parameters mainly affect the \ac{a0}.
In contrast, for the \ac{s0}, only the amplitude changes, except for the reduction in velocity by the increase in wall thickness.

It should be mentioned that all parameters are invariable during the use of the structure and are independent of changing environmental conditions.
Therefore, the core heights and wall thicknesses were set with the dimensions tolerance received from the supplier.
The cell can easily be deformed before bonding to the skin due to the low in-plane stiffness of the core.
Thus for better accuracy, the cell width can be assumed as an average measurement value taken after joining the core with the skin.
The angle of rotation can also be corrected after the components bonding.
\begin{figure}[!htb]
	\begin{center}
		\includegraphics[width=0.95\textwidth]{Chapter_8/core_height}
	\end{center}
	\caption{The signal envelopes for the various \textbf{core heights} in the range 10.5-18.5 \unit{\mm}}
	\label{fig:core_height}
\end{figure}

\begin{figure}[!htb]
	\begin{center}
		\includegraphics[width=0.95\textwidth]{Chapter_8/core_size}
	\end{center}
	\caption{The signal envelopes for the various \textbf{the core size} in the range 5.0-9.0 \unit{\mm}}
	\label{fig:core_size}
\end{figure}

\begin{figure}[!htb]
	\begin{center}
		\includegraphics[width=0.95\textwidth]{Chapter_8/core_thickness}
	\end{center}
	\caption{The signal envelopes for the various \textbf{wall thicknesses} in the range 100-300 \unit{\micro\m}}
	\label{fig:core_thickness}
\end{figure}

\begin{figure}[!htb]
	\begin{center}
		\includegraphics[width=0.95\textwidth]{Chapter_8/core_rotation}
	\end{center}
	\caption{The signal envelopes for the various \textbf{core orientations} in the range \ang{0} - \ang{90}}
	\label{fig:core_rotation}
\end{figure}

\subsection{The \acl{madif}\\ for the double-skin panel}

Finally, computer simulations were conducted to determine the \ac{madif} for a structure with a core between two skins.
The single-skin \ac{fcgm} from the previous analyses was supplemented with a \ac{cfrp} plate and also bonded to the core by the adhesive layer.
Two transducers were attached to the top skin as before.
Due to the impossibility of enlarging the damage in a closed-form structure, no experimental measurements were carried out.
Two damage cases were considered in the simulations: (i) interface elements removed from the upper side (the skin with the sensor attached), and (ii) interface elements removed from the bottom side.

Figure~\ref{fig:madif_2skins_rmsd} presents the \ac{rmsd}-based \ac{madif} for a double-skin panel with the functions obtained for a single-skin for comparison.
Substantial differences between the double- and single-skin panels can be observed.
In addition, the placement of the damage also has influence on the index slope.
\begin{figure}[!htb]
	\begin{center}
		\includegraphics[width=0.95\textwidth]{Chapter_8/MADIF_2skins_rmsd}
	\end{center}
	\caption{Comparison of the \acl{madif} for single-skin and double-skin panels based on \textbf{the \acf{rmsd}}}
	\label{fig:madif_2skins_rmsd}
\end{figure}
Similar changes in the \ac{madif} can be observed for the \ac{cc}, as shown in Figure ~\ref{fig:madif_2skins_cc}, although to a lesser extent than the previous index. 
The \ac{rmsd} values ratio single- to double-skin panel for the most significant damage is about 1.5, while the relevant ratio based on the \ac{cc} is about 1.08.
\begin{figure}[!htb]
	\begin{center}
		\includegraphics[width=0.95\textwidth]{Chapter_8/MADIF_2skins_cc}
	\end{center}
	\caption{Comparison of the \acl{madif} for single-skin and double-skin panels based on \textbf{the \acf{cc}}}
	\label{fig:madif_2skins_cc}
\end{figure}
The obtained \acp{madif} for the double-skin panel meet the conditions for their use in estimating the damage size, i.e. the monotonicity of the function and the significant change in value over the entire range of damage.
\clearpage