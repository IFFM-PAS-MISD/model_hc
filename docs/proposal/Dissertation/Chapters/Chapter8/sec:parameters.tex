%% SECTION HEADER ///////////////////////////////////////////////////////////////////////////////////
\section{The numerical simulation of the \ac{gw} in the \ac{hsc} under various parameters}
\label{sec:parameters}

%% SECTION CONTENT ////////////////////////////////////////////////////////////////////////////////////
In addition to the previous analyses, a parametric study was conducted to determine the effects of various component parameters on \ac{gw} propagation in the \ac{hsc}. The following parameters were taken into account:
\begin{enumerate}
	\item The \ac{pzt} parameters:
	\begin{itemize}
		\item placement in relation to the core cell,
		\item charge constant,
		\item dielectric permittivity,
	\end{itemize}
	\item The \ac{cfrp} and the adhesive layer parameters:
		\begin{itemize}
		\item \ac{cfrp} fibre volume fraction,
		\item adhesive layer thickness,
		\end{itemize}
	\item The core parameters:
	\begin{itemize}
		\item core height and width,
		\item wall thickness,
		\item core rotation angle regarding to wave propagation.
	\end{itemize}
\end{enumerate}

\subsection{The \ac{pzt} parameters}
Computer simulations were conducted to determine the impact of \acp{pzt} placement relative to the core cell.
Seven transducer positions were considered, according to the schematic in Fig.~\ref{fig:pzt_place}(\textbf{b}).
The actuator and sensor position relative to the cell is identical to maintain the same wave propagation distance for each case.

From Fig.~\ref{fig:pzt_place}(\textbf{a}), it can be noticed that the highest amplitude of the \ac{s0} is obtained for the case with the \acp{pzt} placed in the middle of the cell.
The lowest amplitudes were obtained for the cases where the transducer midpoint lay on the core wall.
It is due to the higher stiffness underneath the \ac{pzt}, leading to less displacements.
In the case of the \ac{a0}, the amplitude of the signals for position \#2 and \#5 are much lower than the rest cases.
Since the transducers are permanently fixed to the structure and do not change their position during the monitoring period.
Nonetheless, the \acp{pzt} placement is worth considering to optimise the signals obtained.
\begin{figure}
	\begin{center}
		\includegraphics[width=0.95\textwidth]{Chapter_8/pzt_place}
	\end{center}
	\caption{The \acf{gw} propagation in the \acf{hsc} \textbf{(a)}, sensor responses for \textbf{(b)} different \acf{pzt} placement in relation to the core cell.}
	\label{fig:pzt_place}
\end{figure}

Changes in amplitude under the influence of different values of the piezoelectric charge constant and dielectric permittivity are presented in Fig.~\ref{fig:pzt_d} and ~\ref{fig:pzt_eps}, respectively.
Both parameters have a significant effect on the signal amplitude both the \ac{s0} and \ac{a0}.
The values do not just change with ambient temperature but also degrade over service time \cite{barzegar2001aging, deangelis2006p2o}.
Therefore, I suggest that a sensor self-diagnostic tool should be included in the \ac{shm} system. 
An \ac{emi}-based method could be a good solution as it was suggested for self-diagnostic of damaged transducers by Jiang et al. \cite{jiang2021electromechanical}.

\begin{figure}
	\begin{center}
		\includegraphics[width=0.95\textwidth]{Chapter_8/pzt_d}
	\end{center}
	\caption{The signal envelopes for different piezoelectric charge constants.}
	\label{fig:pzt_d}
\end{figure}

\begin{figure}
	\begin{center}
		\includegraphics[width=0.95\textwidth]{Chapter_8/pzt_eps}
	\end{center}
	\caption{The signal envelopes for different dielectric permittivities.}
	\label{fig:pzt_eps}
\end{figure}

\subsection{The \ac{cfrp} skin and the adhesive layer parameters}

The effect of the skin properties on wave propagation in the \ac{hsc} was analysed for different carbon fibre volume fractions in the composite.
This parameter affects the mode velocity due to the change in effective modulus of elasticity and density of the plate.
A small change in modulus of elasticity leads to significant signal differences within the interference of wave reflections as it can be observe in Fig.~\ref{fig:skin_volume}.
It affects the \ac{madif} determination since the full-length signal is taken into account.

\begin{figure}
	\begin{center}
		\includegraphics[width=0.95\textwidth]{Chapter_8/skin_volume}
	\end{center}
		\caption{Sensor responses in the \acf{hsc} for various fibre volume fractions in the range 45-50\%.}
	\label{fig:skin_volume}
\end{figure}

The effect of the adhesive layer thickness is presented in Fig.~\ref{fig:adhesive_thickness}.
The greater the adhesive thickness, the slower both modes propagate.
There is also a noticeable decrease in the \ac{a0} amplitude, while the \ac{s0} amplitude barely changes.
It is because the \ac{a0} displacements are mainly out-of-plane, so more energy of the wave leaks into the adhesive, and it is attenuated in low-stiffness material.

\begin{figure}
	\begin{center}
		\includegraphics[width=0.95\textwidth]{Chapter_8/adhesive_thickness}
	\end{center}
	\caption{Sensor responses in the \acf{hsc} for various adhesive thicknesses in the range 200-500 \(\mu\)m.}
	\label{fig:adhesive_thickness}
\end{figure}

\subsection{The core parameters}
In the study of the core geometry influence on wave propagation in \ac{hsc}, four parameters were considered:
\begin{itemize}
	\item core height \(g=[10.5,\,12.5,\,14.5,\,16.5,\,18.5]\) \unit{\mm};
	\item core width \(l_1=[5.0,\,6.0,\,7.0,\,8.0,\,9.0]\) \unit{\mm};
	\item wall thickness \(w_c=[100,\,150,\,200,\,250,\,300]\) \unit{\micro\m};
	\item core rotation angle regarding to wave propagation [\ang{0}, \ang{15}, \ang{30}, \ang{45}, \ang{60}, \ang{75}, \ang{90}].
\end{itemize}

From the signal envelopes shown in Fig.~\ref{fig:core_height}, \ref{fig:core_size}, \ref{fig:core_thickness} and \ref{fig:core_rotation} it can be seen that the analysed parameters mainly affect the \ac{a0}. 
In contrast, for the \ac{s0}, only the amplitude changes, except for the reduction in velocity by the increase in wall thickness.

It should be mentioned that all parameters are invariable during the use of the structure and are independent of changing environmental conditions.
Therefore, the core heights and wall thicknesses are set with the dimensions tolerance received from the supplier.
The cell can easily be deformed before bonding to the skin due to the low in-plane stiffness of the core.
Thus for better accuracy, the cell width can be assumed as an average measurement value taken after joining the core with the skin.
The angle of rotation can also be corrected after the components bonding.
\begin{figure}
	\begin{center}
		\includegraphics[width=0.95\textwidth]{Chapter_8/core_height}
	\end{center}
	\caption{Sensor responses in the \acf{hsc} for the various core heights in the range 10.5-18.5 \unit{\mm}.}
	\label{fig:core_height}
\end{figure}

\begin{figure}
	\begin{center}
		\includegraphics[width=0.95\textwidth]{Chapter_8/core_size}
	\end{center}
	\caption{Sensor responses in the \acf{hsc} for the various cell widths in the range 5.0-9.0 \unit{\mm}.}
	\label{fig:core_size}
\end{figure}

\begin{figure}
	\begin{center}
		\includegraphics[width=0.95\textwidth]{Chapter_8/core_thickness}
	\end{center}
	\caption{Sensor responses in the \acf{hsc} for the various wall thicknesses in the range 100-300 \unit{\micro\m}.}
	\label{fig:core_thickness}
\end{figure}

\begin{figure}
	\begin{center}
		\includegraphics[width=0.95\textwidth]{Chapter_8/core_rotation}
	\end{center}
	\caption{Sensor responses in the \acf{hsc} for the various core orientations.}
	\label{fig:core_rotation}
\end{figure}

\subsection{The \ac{madif} for the double-skin \ac{hsc}}

Finally, computer simulations were conducted to determine the \ac{madif} for a structure with a core between two skins.
The single-skin \ac{fcgm} from the previous analyses was supplemented with a \ac{cfrp} plate and also bonded to the core by the adhesive layer.
Two transducers were attached to the top skin as before.
Due to the impossibility of enlarging the damage in a closed-form structure, no experimental measurements were carried out.
Two damage cases were considered in the simulations: (i) interface elements removed from the upper side (the skin with the sensor attached), and (ii) interface elements removed from the bottom side.

Fig.~\ref{fig:madif_2skins}\textbf{(a)} presents the \ac{rmsd}-based \ac{madif} for a double-skin panel with the functions obtained for a single-skin for comparison.
Substantial differences between the double- and single-skin panels can be observed.
In addition, the placement of the damage has also influence on the index slope.
Similar changes in the \ac{madif} can be observed for \ac{cc}, as shown in Fig. ~\ref{fig:MADIF_2skins}\textbf{(b)}.
Although to a lesser extent than the previous index. 
The \ac{rmsd} values ratio single- to double-skin panel for the most significant damage is about 1.5, while the relevant ratio based on \ac{cc} is about 1.08.
\begin{figure}
	\begin{center}
		\includegraphics[width=0.95\textwidth]{Chapter_8/MADIF_2skins}
	\end{center}
	\caption{Comparison of the \acf{madif} for single-skin and double-skin panels based on \textbf{(a)}the \acf{rmsd} and \textbf{(b)} the \acf{cc}.}
	\label{fig:madif_2skins}
\end{figure}