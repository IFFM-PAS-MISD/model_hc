% Set page numbering to arabic the first time we commence a chapter.
% This is required to get the page numbering correct.

% Note that the text in the [] brackets is the one that will
% appear in the table of contents, whilst the text in the {}
% brackets will appear in the main thesis.

%% CHAPTER HEADER /////////////////////////////////////////////////////////////////////////////////////
\chapter[Problem Statement]{Problem Statement}
\label{ch:problem}

%% CHAPTER INTRODUCTION ////////////////////////////////////////////////////////////////////////////////////

%% INCLUDE SECTIONS ////////////////////////////////////////////////////////////////////////////////////

Composite materials are used as structural components whose failure can have catastrophic consequences.
Although \ac{gw} based methods are promising for damage identification and localization, they have not been widely reported to estimate the failure size.
A better understanding of the effect of damage on elastic wave propagation in \acp{hsc} will allow the development of robust and practical tools to evaluate this structure.

The principal aim of this dissertation was to propose a new approach for the severity of damage identification in the \ac{hsc} employing the \ac{pzt} sensors.
The essence of the method is the determination of damage influence function on characteristic parameters of the propagating waves in the structure.
This function was defined using a numerical model developed by the \ac{sem}.

Initially, the model is prepared for the pristine sample under various operating conditions such as ambient temperature.
Then parametric simulations for different damage scenarios are performed.
The \ac{madif} is determined based on the obtained results. 
This function determines the flaw size depending on the \acp{pzt} response.
Damage is considered to be a debonding between the core and the skin. 
The following objectives of the dissertation are:
\begin{itemize}
	\item Develop a robust and efficient numerical model of the propagating \ac{gw} in the \ac{hsc} under varying ambient temperatures.
	\item Validate the model experimentally.
	\item Determine the \ac{madif} to define damage influence on the propagating waves.
	\item Investigate the structure experimentally under varied temperature conditions.
	\item Propose a framework of damage detection based on the \ac{madif}.
\end{itemize}

The imposed objectives lead to the proof of the thesis that it is possible to determine the function of damage severity in \acp{hsc} by numerical simulations.

Chapter \ref{ch:method} presents the methods for developing a model-assisted damage severity assessment scheme.

Chapter \ref{ch:sem} gives a theoretical background of the \ac{sem} for \ac{gw} propagation.
It includes mass, stiffness and damping matrices for 2D and 3D formulation; interface coupling algorithm; time integration scheme; and parallel implementation for GPU calculation.

Chapter \ref{ch:simulation} provides the details of the sample configuration for the full geometry and homogenized core models. The individual sample component grids, the signal parameters, and the damage model are presented.

The simulations and experimental validation results are presented in Chapter \ref{ch:validation}.

Chapter \ref{ch:tempEffects} includes the analysis of the GW propagation under variable temperature conditions.

The crucial part of the dissertation appears in Chapter \ref{ch:severity}. Based on the simulations performed and their validation experimentally, the function of the damage size effect on the elastic wave propagation in \ac{hsc} is revealed, termed \ac{madif}.

The conclusion of the dissertation and final remarks are provided in Chapter \ref{ch:conclusions}.

