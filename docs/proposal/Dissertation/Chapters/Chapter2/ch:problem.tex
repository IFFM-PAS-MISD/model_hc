% Set page numbering to arabic the first time we commence a chapter.
% This is required to get the page numbering correct.

% Note that the text in the [] brackets is the one that will
% appear in the table of contents, whilst the text in the {}
% brackets will appear in the main thesis.

%% CHAPTER HEADER /////////////////////////////////////////////////////////////////////////////////////
\chapter[Problem Statement]{Problem Statement}
\label{ch:problem}

%% CHAPTER INTRODUCTION ////////////////////////////////////////////////////////////////////////////////////

%% INCLUDE SECTIONS ////////////////////////////////////////////////////////////////////////////////////

Composite materials are used as structural components whose failure can have catastrophic consequences.
Although \ac{gw}-based methods are promising for damage detection and localisation, they have not been widely reported to estimate damage size.
A better understanding of the effect of damage on elastic wave propagation in the \acp{hsc} may allow the development of robust and practical tools to evaluate this structure.

The principal aim of this dissertation was to propose a new approach for damage severity identification in \ac{hsc} employing the \acp{pzt}.
The essence of the method was the determination of damage influence function on characteristic parameters of the propagating waves in the structure.
This function was defined using a numerical model developed using the \ac{sem}.

Initially, the model was prepared for the healthy sample.
Then parametric simulations for various damage size were performed.
The \ac{madif} was determined based on the obtained results. 
This function determines the flaw size depending on the \acp{pzt} response.
For the purposes of the research, debonds of the core and the skin were considered as damage.
The \ac{madif} was also determined at various ambient temperatures.

The objectives of the dissertation were as follows:
\begin{itemize}
	\item develop a robust and efficient numerical model of the propagating \ac{gw} in \ac{hsc} with the temperature as an additional parameter
	\item validate the model experimentally
	\item determine the \ac{madif} to define damage influence on the propagating waves
	\item investigate the \ac{madif} under various ambient temperatures
	\item study the effects of various \ac{hsc} parameters on the \ac{madif}.
\end{itemize}

The presented objectives leaded to the thesis formulated as follows:
\begin{thesis*}  
	Model-assisted analysis of guided wave propagation is an effective tool for determining the damage severity in honeycomb sandwich composites.
\label{thesis}
\end{thesis*}

Chapter~\ref{ch:method} presents the methods for developing a model-assisted damage severity assessment scheme.

Chapter~\ref{ch:sem} gives a theoretical background of the \ac{sem} for \ac{gw} propagation.
It includes the derivation of mass, stiffness and damping matrices for 2D and 3D formulation; interface coupling algorithm; time integration scheme; and parallel implementation for GPU calculation.

Chapter~\ref{ch:simulation} details the sample configuration for the \ac{fcgm} and the \ac{hcgm} in the \ac{sem} framework.
The damage model, signal parameters, and the individual component meshing process are all presented. Additionally, temporal and spatial convergence tests for the developed models are presented in the Chapter.

Chapter~\ref{ch:validation} includes the results of experimental validation for the individual model of the \ac{pzt} and the both model of \ac{hsc}.

The crucial part of the dissertation appears in Chapter \ref{ch:severity}.
Based on the performed simulations and their experimental validation, the function of the damage size effect on the elastic wave propagation in \ac{hsc} was established, termed \ac{madif}.

Chapter~\ref{ch:tempEffects} includes the analysis of \ac{gw} propagation under variable temperature conditions. Moreover, a parametric study is presented that determines the effect of various sample parameters on wave propagation in \ac{hsc}.

A summary of the dissertation and concluding remarks can be found in Chapter~\ref{ch:summary}.
