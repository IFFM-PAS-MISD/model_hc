% Note that the text in the [] brackets is the one that will
% appear in the table of contents, whilst the text in the {}
% brackets will appear in the main thesis.

%% CHAPTER HEADER /////////////////////////////////////////////////////////////////////////////////////
\chapter[Summary]{Summary}
\label{ch:summary}

This dissertation develops a model-assisted method for identifying the severity of mechanical damage in a honeycomb sandwich composite. 
For this purpose, the guided wave method was used, based on the change in elastic wave propagation under the influence of their interaction with damage in the monitored structure.
A phenomenon called wave leakage in sandwich panels results in the transmission of the wave into the structure core, where wave energy is attenuated.
In the case of damage such as disbonds of the core from the skin, the phenomenon does not occur.
Thus, a signal recorded by a sensor is distorted regarding a healthy one.

The effect of damage size on wave propagation was done using computer simulations, resulting in a model-assisted identification function.
Due to the high complexity of the structure under analysis, an accurate and efficient model had to be used.
It was achieved by engaging the spectral element method, one of the most robust tools for modelling elastic wave propagation.
Chapter \ref{ch:sem} presents a full implementation of the method for the honeycomb sandwich composite.
One- and two-dimensional spectral elements are used in the model, and a multi-physics model of piezoelectric transducers is implemented for the excitation and recording of guided waves.
This approach required the development of an interface to connect all the elements.
The dissertation presents a new method to determine the non-matching interface elements using the shape function of the spectral elements.

Two core models were analysed, i.e. (i) the full core geometry model and (ii) the homogenised core geometry model.
A rectangular core-skin disbond of variable width and fixed length was taken as the damage placed in the centre of the panel.
The defect was modelled by removing (i) the core cells and (ii) the interface elements.
Then, after the model validation presented in chapter \ref{ch:validation}, computer simulations were performed for different damage sizes and varied excitation signal frequencies.

Obtained signals were proceeded and then used to determine several damage indices.
It turned out that only two indices satisfied the selection criterion, i.e. monotonic over the full range of damage size and the magnitude of the function value.
These two functions are the indices based on root mean square deviation and correlation coefficient for 100 kHz signals and fulfil the criteria for all models.
After comparing model-assisted damage identification functions with the experimental results, it appears that the full core geometry model is more accurate than the homogenised one.
That model with the index based on root mean square deviation achieved an excellent agreement with the experimental investigation.

In Chapter \ref{ch:tempEffects}, an analysis of the effect of ambient temperature on the function is presented.
A parametric study followed this to determine the effect of varied components parameters on the elastic wave propagation in honeycomb sandwich composite. 
It turns out that some parameters only affect the determination of the reference signals, and some depend on the conditions prevailing during the structure inspection.
It is recommended that an effective tool be developed to work out the current parameters of the structure.

Major contribution of the dissertation are:
\begin{itemize}
	\item develop of the full core geometry model based on spectral element method;
	\item develop of the non-matching interface elements,
	\item determination of the model-assisted damage identification function for honeycomb sandwich composite,
	\item time-defendant study on the model-assisted damage identification function
	\item optimisation of the time integration algorithm for parallel computation on graphics processing unit.
\end{itemize}

The dissertation proved the thesis validity: model-assisted analysis of guided wave propagation is an effective tool for determining the damage severity in honeycomb sandwich composites.
The results achieved in the dissertation encourage further work on this topic.
Namely, the analysis of functions for damage with different shapes, e.g. circles or ellipses, different positions.
