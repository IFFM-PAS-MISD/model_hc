%% SECTION HEADER /////////////////////////////////////////////////////////////////////////////////////
\section{Structural Health Monitoring}
\label{sec:scm}

%% SECTION CONTENT ////////////////////////////////////////////////////////////////////////////////////
\Ac{shm} is the process of implementing an advanced damage identification strategy for structural or mechanical systems \cite{farrar2007introduction}.
The \ac{shm} systems usually consist of a sensor network, a \ac{dau}, and a central processor.
The \ac{dau} is responsible for collecting the data measured by the sensor network and the intact structure data if used \ac{shm} technique requires it.
A central unit then determines the current state of the structure through signal processing and statistical classification.
The implementation of \ac{shm} aims to extend the safe life of the monitored system, or usage of lightweight materials, which leads to cost reduction in production and operation.
For example, composites and adhesive bonding techniques reduces the aircraft's overall weight, reducing fuel consumption \cite{scelsi2011potential}.
\ac{shm} is most commonly found in structures, such as aerospace, civil and mechanical engineering, where damage can have catastrophic consequences.

Rytter, in his dissertation \cite{rytter1993vibrational}, classified the \ac{shm} system advancement into the following four levels:
\begin{itemize}
	\item[] \textbf{Level 1}: Detection.
	\item[] \textbf{Level 2}: Localization.
	\item[] \textbf{Level 3}: Assessment.
	\item[] \textbf{Level 4}: Consequence.
\end{itemize}
The first level determines if any adverse change in the geometric has occurred or material characteristics of the system. The second level leads to the localization of the damage.
The third and fourth level systems determine the size of the flaw and decide whether any maintenance is necessary, respectively.
The existence and location of faults can be defined in unsupervised learning mode by taking a threshold value for a measurable, damage-sensitive system feature. The threshold should be compensated depending on the prevailing operational and environmental conditions.
In contrast, damage size is determined in supervised learning mode based on an analytical model or data extracted experimentally from the structure \cite{worden2007fundamental}.