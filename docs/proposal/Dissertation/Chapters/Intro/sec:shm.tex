%% SECTION HEADER /////////////////////////////////////////////////////////////////////////////////////
\section{Structural Health Monitoring}
\label{sec:scm}

%% SECTION CONTENT ////////////////////////////////////////////////////////////////////////////////////
\Ac{shm} is the process of implementing an advanced damage identification strategy for structural or mechanical systems \cite{farrar2007introduction}.
\ac{shm} is most commonly found in structures, such as aerospace, civil and mechanical engineering, where damage can have catastrophic consequences.
The state of the system is determined by analyzing periodically obtained measurements.
\ac{shm} implementation aims to extend the safe life of equipment, use lightweight materials, and reduce manufacturing and operating costs.
For example, the use of composites and adhesive bonding techniques reduces the aircraft's overall weight, thereby reducing fuel consumption \cite{scelsi2011potential}.

The fundamentals of \ac{shm} classify damage identification into four levels \cite{rytter1993vibrational}:
\begin{itemize}
	\item[] \textbf{Level 1}: Detection
	\item[] \textbf{Level 2}: Localization,
	\item[] \textbf{Level 3}: Assessment,
	\item[] \textbf{Level 4}: Consequence.
\end{itemize}
The first level determines if there is an adverse change in the geometric or material characteristics of the system.
Detection and localization can be realized by an unsupervised learning mode, while the third level requires knowledge of the system to assess the damage severity \cite{worden2007fundamental}. 


The \ac{gw} propagation method is a high-potential approach in \ac{shm} for
damage detection in \acp{scs} \cite{mustapha2011assessment, sikdar2016guided, sikdar2016ultrasonic,radzienski2016assessment, yu2019core}.
\ac{gw} are mechanical waves that propagate in a bounded
elastic medium, e.g., bars, beams, rods, plates and shells.
An excitation and sensing of the
GW can be realized by the lightweight and inexpensive \acp{pzt} [6].
The compact \ac{pzt} can be surface-bonded to the inspected structure or even embedded between the composite plies so that the measurements can be conducted \textit{in situ}.
Among numerous \ac{gw}-based techniques developed for damage detection and localization, the most popular are pitch--catch \cite{ihn2008pitch, sikdar2017structural}, pulse--echo \cite{guo1993interaction, kudela2008damage}, phase array \cite{lu2006crack, ostachowicz2008elastic}, and time-reversal mirror \cite{fink1992time, eremin2016analytically}.
For damage identification, some of them require a baseline to be determined.
Due to the cost and time-consuming experimental investigation is an inefficient approach to obtain references.

Excitation and sensing of the \ac{gw} can be realized by the lightweight and inexpensive \acp{pzt} \cite{giurgiutiumicromechatronics}.
The compact \ac{pzt} can be surface--bonded to the inspected structure or even embedded between the composite plies so that the measurements can be conducted \textit{in situ}.
The \acp{pzt} generate high forces with broadband frequency, so methods based on \ac{gw} can detect various damage types of different sizes in a large inspected area \cite{su2006guided}. Moreover, certain algorithms do not require a baseline model.
Among numerous techniques developed for damage detection and localization, the most popular are pitch--catch \cite{ihn2008pitch, sikdar2017structural}, pulse--echo \cite{guo1993interaction, kudela2008damage}, phase array \cite{lu2006crack, ostachowicz2008elastic}, and time--reversal mirror \cite{fink1992time, eremin2016analytically}.
