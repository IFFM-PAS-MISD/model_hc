%% SECTION HEADER /////////////////////////////////////////////////////////////////////////////////////
\section{Challenges in Damage Assessment in the \ac{hsc}}
\label{sec:challenges}

%% SECTION CONTENT ////////////////////////////////////////////////////////////////////////////////////

The assessment of damage magnitude in structural materials can generally be performed in a supervised learning mode \cite{worden2007fundamental}.
In the case of \ac{hsc}, many factors affect the \ac{di} magnitude, such as damage localization, components material properties and dimensions, the sensor position relative to the core cell and the boundary conditions.
Considering all the factors, the determination of DI by experimental means becomes very complicated, expensive and time-consuming.
Therefore, numerical analysis and computer simulations become the only useful tool to achieve the goal.

The most popular found in the literature model for \ac{gw} propagation in \ac{hsc} is the \ac{fem} developed by commercial software.
Although the numerical results agree with the experimental, the models have limited in size, or the core has to be homogenized to perform simulations in a reasonable time.