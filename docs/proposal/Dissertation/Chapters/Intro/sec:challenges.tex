%% SECTION HEADER /////////////////////////////////////////////////////////////////////////////////////
\section{Challenges in \acl{gw} propagation modelling for damage assessment in \acl{hsc}}
\label{sec:challenges}

%% SECTION CONTENT ////////////////////////////////////////////////////////////////////////////////////
Assessing the damage severity in structural materials requires a database of the damage influence on system response \cite{worden2007fundamental}.
In the case of \ac{hsc}, many factors affect the \ac{di} magnitude, such as damage localisation, material properties and dimensions, the sensor position relative to the core cell and the boundary conditions.
Determining \ac{di} by experimental means becomes very complicated, expensive and time-consuming, considering all the factors.
Therefore, numerical analysis and computer simulations become the only practical tool to achieve the goal.

The most common \ac{hsc} numerical model used in analysing \acp{gw} and \ac{emi} found in the literature is based on the \ac{fem}.
Although the numerical results are close to the experimental results, the models have some limitations in performing the simulations with reasonable computation time and operating memory consumption.
In order to alleviate these limitations, the techniques are employed as follows:
\begin{itemize}
\item reduction of the sample dimensions \cite{hosseini2013numerical, tian2015wavenumber},
\item homogenisatin of the core properties \cite{catapano2014multi, zhou2020debonding},
\item a simplified \acl{2d} model based on a cross-section of the panel~ \cite{li2019detection},
\item omission of an adhesive layer \cite{mustapha2013non}.
\end{itemize}

The time and memory consumption of the \ac{fem} simulations is due to the high spatial resolution needed to converge the numerical results.
When first-order elements are used, up to 20 nodes per the shortest wavelength of interest is recommended by Moser et al. \cite{moser1999modeling}.
At a time when the operational capabilities of personal computers were severely limited, several methods derived from classic \ac{fem} were developed to increase the efficiency of calculations.
The \ac{bem} has been widely used for wave propagation in infinite or semi-infinite areas \cite{brebbia1984boundary}.
This method uses the fundamental solution of a \ac{pde} with the approximation only at the boundary of the domain.
The \ac{fdm} also has found application in elastic wave modelling \cite{delsantoO1992connection}.
The solution of the \ac{pde} is realised by the Taylor expansion with the arbitrary number of terms that determines the order of accuracy \cite{willberg2015simulation}.
The main disadvantage of this method is the loss of numerical stability for the structure with varying material properties.

To avoid this drawback, Delsanto et al. developed the \ac{lisa} with the combination of a sharp interface model to connect to different domains \cite{delsantoO1992connection}. This method is the extension of the \ac{fdm} in which iteration equations are obtained directly from heuristic consideration \cite{willberg2015simulation}.
The \ac{lisa} is highly numerically efficient because local interactions between elements are directly transferred for numerical calculations.

However, no usage of the \ac{bem}, \ac{fdm} and \ac{lisa} in \ac{hsc} modelling has been documented in the literature, supposedly due to the complex geometry of the core.
Realistic shape of the core can be implemented while keeping numerical efficiency using a technique based on high-order elements.
The implementations of this technique have been developed in recent years, achieving convergence even at six nodes in case of \ac{gw} propagating in isotropic plates \cite{willberg2012comparison}.
One of the implementations is a method based on Lagrange polynomials as a shape function and the \ac{gll} for integration scheme, termed the \ac{sem}.
The \ac{sem} was initially developed by Patera \cite{patera1984spectral} for the numerical solution of the fluid flow in a channel.
The method has also been successfully employed for acoustic wave propagation in geological structures \cite{seriani1994spectral, komatitsch2000simulation} and \ac{gw} propagation in engineering elements \cite{kudela2007wave, ostachowicz2011guided, rucka2010experimental,rekatsinas2017cubic}.
The \ac{sem} can be applied to complex structures, e.g. stiffened panels \cite{schulte2011simulation, lonkar2014modeling}.
The versatility of the method is also related to the use of hybrid elements in combination with \ac{fem} formulation \cite{ha2009optimizing} and non-linear issues \cite{yu2020time, li2021hybrid}.

Due to the fast convergence and flexibility of the \ac{sem}, Kudela \cite{kudela2016parallel} applied the method to the \ac{hsc} model with the \ac{fcgm}.
However, the author used \ac{3d} elements to model the core walls resulting in a huge number of \acp{dof}.
The model size could be reduced if the shell element replaced the solid one.
In addition, the surface-mounted \ac{pzt} is omitted in the above model, and a concentrated force is used as the disturbance source.
The sensor mesh would have to coincide with the plate mesh or use the coupling between both meshes to include the \acp{pzt} in the simulation.
Such coupling can be realised using an interface based on Lagrange multipliers proposed by Farhat and Roux for domain decomposition in \ac{fem} \cite{farhat1991method}.
Ashwin et al. implemented the interface for the \ac{sem} but did not adopt it to non-matching grids \cite{ashwin2014formulation}, which is required to generalise the model.
Therefore, one of the goals of the dissertation was to implement a non-matching grid approach for the \ac{sem}.

Besides a computationally efficient model, an important factor affecting the speed of computer simulations is the hardware on which they are performed.
Recently, the use of \ac{gpu}-equipped workstations in numerical computation has grown.
However, to employ the \ac{gpu} for calculations, the time integration algorithm must be prepared for parallel processing.
The implementation of the algorithm for the \ac{sem} was presented by Kudela \cite{kudela2016parallel} and by Paćko et al. for the \ac{lisa} \cite{packo2012lamb}.
According to \cite{kudela2016parallel}, the multi-core architecture of the \ac{gpu} enables simultaneous vector operations, making simulations over 14 times faster than those performed by a \ac{cpu}.
Determining the effect of damage size for models with a large number of \ac{dof} should be done using the \ac{gpu} to perform a series of simulations in a reasonable amount of time.