%% SECTION HEADER /////////////////////////////////////////////////////////////////////////////////////
\section{Challenges in Damage Assessment in the \ac{hsc}}
\label{sec:challenges}

%% SECTION CONTENT ////////////////////////////////////////////////////////////////////////////////////
The assessment of the damage magnitude in structural materials requires the development of a database of the effect of damage on system response \cite{worden2007fundamental}.
In the case of \ac{hsc}, many factors affect the \ac{di} magnitude, such as damage localization, material properties and dimensions, the sensor position relative to the core cell and the boundary conditions.
Considering all the factors, the determination of DI by experimental means becomes very complicated, expensive and time-consuming.
Therefore, numerical analysis and computer simulations become the only practical tool to achieve the goal.

The most common \ac{hsc} numerical model used in analysing \acp{gw} and \ac{emi} found in the literature is the model based on the \ac{fem}.
Although the numerical results are consistent with the experimental results, the models have some limitations in performing the simulations in a reasonable computation time and operating memory consumption.
The improvements include:
\begin{itemize}
\item reduction of the sample dimensions \cite{hosseini2013numerical, tian2015wavenumber};
\item homogenisation of the core properties \cite{catapano2014multi, zhou2020debonding};
\item a simplified \ac{2d} model based on a cross-section of the panel \cite{li2019detection};
\item neglect of an adhesive layer \cite{mustapha2013non}.
\end{itemize}
The time and memory consumption of the FEM simulations is due to the high spatial resolution needed to converge the numerical results.
When first-order elements are used, up to twenty nodes per the shortest wavelength of interest is recommended by Moser et al. \cite{moser1999modeling}.
Therefore, implementations with higher-order elements have been developed in recent years, achieving convergence even at six nodes \cite{willberg2012comparison}.
One is a method based on Lagrange polynomials as a shape function and \ac{gll} for integration scheme, termed \ac{sem}.
The \ac{sem} has initially developed for the numerical solution of the fluid flow in a channel by Patera \cite{patera1984spectral}. The method has also been successfully employed for elastic wave propagation by many researchers \cite{seriani1994spectral, kudela2007wave, ostachowicz2011guided, rucka2010experimental, schulte2011simulation, lonkar2014modeling, rekatsinas2017cubic, yu2020time, li2021hybrid}.

Due to the fast convergence and flexibility of the \ac{sem}, Kudela applied the method to model \ac{hsc} with full core geometry.
Despite the small size of the plate ($179 \times 159$ mm), the model has more than 1.5 million \acp{dof} because the skins and the core are composed of 3D elements.
The model size could be reduced if the shell element replaced the solid one.
In addition, the surface-mounted \ac{pzt} is omitted in the above model, and a concentrated force is used as the disturbance source.
The sensors mesh would have to coincide with the plate mesh or use the coupling between both meshes to include the \acp{pzt} in the simulation.
Such coupling can be realized using an interface based on Lagrange multipliers proposed by Farhat and Roux for domain decomposition in \ac{fem} \cite{farhat1991method}.
Ashwin et al. implemented the interface for the \ac{sem} but did not adopt it to non-matching grids \cite{ashwin2014formulation}, which is required for model generalization.

The efficient model and the computational hardware play a significant role in the simulation speed. 
Kudela presented the \ac{sem} algorithm for parallel calculation on the \ac{gpu} in the paper mentioned above.
The multi-core architecture of the \ac{gpu} enables simultaneous vector operations, making simulations over 14 times as fast as those performed by a \ac{cpu}.
Therefore, using the card will make it more convenient to perform simulations of models with numerous \acp{dof} to determine the effect of damage size on system response.
