%% SECTION HEADER /////////////////////////////////////////////////////////////////////////////////////
\section{Piezoelectric transducers}
\label{sec:PZT}

%% SECTION CONTENT ////////////////////////////////////////////////////////////////////////////////////
The piezoelectric phenomenon is the generation of an electrical charge on the surface of materials under mechanical deformation in crystalline materials with no inversion symmetry.
The magnitude of the generated charge is proportional to the strain and the direction of polarization.
Those materials also exhibit the opposite effect: a change in size due to an applied electric field.
Piezoelectric materials are widely used in engineering as electroacoustic transducers, high voltage generators and power sources, energy harvesters, micro motors and actuators.
\begin{figure}[H]
	\begin{center}
		\includegraphics[width=0.95\textwidth]{Intro/PZTs}
	\end{center}
	\caption{Various types of piezoelectric transducers (\textbf{a}) circular discs, (\textbf{b}) circular array of the transducers, (\textbf{c}) Smart Layer\textsuperscript{\tiny\textregistered} sensors - piezoelectrics embedded into dielectric film manufactured by Acellent Technologies, Inc.}
	\label{fig:piezo}
\end{figure}
The \acp{pzt}, the acronym derived from the chemical formula of the most commonly used piezoelectric ceramic, i.e. Pb[Zr\(_x\)Ti\(_{1-x}\)]O\(_3\) (lead zirconate titanate), are lightweight, various size and shape structures.
Examples of ready-to-use \ac{pzt} are shown in Figure~\ref{fig:piezo}.
They can be permanently mounted on the structure surface, embedded within the material, or even be a smart composite material.
In the \ac{shm}, they are mainly used in elastic wave propagation, modal analysis, and the \ac{emi} methods.