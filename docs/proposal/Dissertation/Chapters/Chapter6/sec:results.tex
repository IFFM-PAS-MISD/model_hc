%% SECTION HEADER /////////////////////////////////////////////////////////////////////////////////////
\section{Results}
\label{sec:resuls}

%% SECTION CONTENT ////////////////////////////////////////////////////////////////////////////////////
The snapshots for the pristine and the damaged sample are shown in \mbox{Figures~\ref{fig:wavefield} and \ref{fig:wavefield_dam5}}, respectively.
One can observe the wave reflections in the core cells for experimental measurements and the present model. 
Additional, the front of the incident wave is distorted for the measurements of 125 and 150 kHz.
The wavefront distortion in the present model is observed in the full range of frequency. 

\vspace{-6pt}
\begin{figure}[H]
	%	\begin{center}
	\includegraphics[width=0.95\linewidth]{Chapter_6/fullfield}
	%	\end{center}
	\caption{The top surface out of plane particle velocity snapshots in time 100 \(\mu\)s for (\textbf{a}) the experimental results obtained by using \ac{sldv}, (\textbf{b}) the present model and (\textbf{c}) the homogenized model in the pristine~sample.}
	\label{fig:wavefield}
\end{figure}
\begin{figure}[H]
	%	\begin{center}
	\includegraphics[width=0.98\linewidth]{Chapter_6/fullfield_dam5}
	%	\end{center}
	\caption{The top surface out of plane particle velocity snapshots in time 100~\(\mu\)s for (\textbf{a}) the experimental results obtained by using \ac{sldv}, (\textbf{b}) the present model and (\textbf{c}) the homogenized model in the sample with 90 mm damage.}
	\label{fig:wavefield_dam5}
\end{figure}

Such  effects are not noticeable in the simplified model because the wave propagates smoothly through the structure.
The wavefront improvement in the experiment and the present model is noticeable in the undamaged region and marked by the red curves in Figure~\ref{fig:wavefield_dam5}.
This is the effect of a lack of reflection with the core.