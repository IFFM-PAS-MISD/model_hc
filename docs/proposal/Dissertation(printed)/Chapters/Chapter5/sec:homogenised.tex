%% SECTION HEADER /////////////////////////////////////////////////////////////////////////////////////
\section{\Acl{hcgm}}
\label{sec:homogenised}

%% SECTION CONTENT ////////////////////////////////////////////////////////////////////////////////////
Section \ref{sec:modelling} contains several examples of the successful application of the model to numerical analysis of \ac{gw} propagation and damage localisation in \ac{hsc}.
Its unquestionable advantage is the simplified component mesh, reducing operating memory resources.
In the dissertation, comparative studies between the \ac{hcgm} and the \ac{fcgm} were conducted to assess the simplified model effectiveness in estimating damage size.

In the \ac{hcgm}, the values of the material constants of the panel core were calculated according to the method presented by Malek and Gibson \cite{malek2015effective}.
This model is an extension of the theoretical analysis of Gibson et al. \cite{gibson1982mechanics}, considering the different geometry of the cell and the nodes at the intersection of the vertical and inclined walls.

The comprehensive formulation of the stiffness matrix components is given in Appendix~\ref{app:eff_properties} and the effective mechanical properties are gathered in Table \ref{tab:properties_eff}.
The properties for other structures, i.e., the skin, the epoxy adhesive, the cyanoacrylate glue, and the sensors remained unchanged.
The core element has \(6 \times 6 \times 4\) nodes, and the mesh coincides with the skin mesh.
The elements of the other structures are the same as described in the previous Section.