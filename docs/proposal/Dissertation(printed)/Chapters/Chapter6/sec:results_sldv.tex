%% SECTION HEADER /////////////////////////////////////////////////////////////////////////////////////
\section{Results of \acl{hsc} validation with the \acl{sldv} setup}
\label{sec:resuls_sldv}
%% SECTION CONTENT ////////////////////////////////////////////////////////////////////////////////////
Figures~\ref{fig:fullfield_50_0}, \ref{fig:fullfield_100_0} and \ref{fig:fullfield_150_0} present the full wavefield in the healthy sample.
The experimental measurements and the \ac{fcgm} snapshots show that the wavefront distortion is rising with the frequency.
Because the wavelength decreases as the frequency increases, a higher frequency signal is more likely to induce wave reflections from the core walls.
This effect can not be observed in the case of the \ac{hcgm}.

\begin{figure}[!hbt]
	\begin{center}
		\includegraphics[width=0.95\textwidth]{Chapter_6/fullfield_50_0}
	\end{center}
	\caption{The top surface out-of-plane velocity snapshots for (\textbf{a}) the \acf{fcgm}, (\textbf{b}) the experimental results obtained by \acf{sldv}, and (\textbf{c}) the \acf{hcgm} in the \textbf{healthy~sample at 50 kHz}}
	\label{fig:fullfield_50_0}
\end{figure}
\begin{figure}[!hbt]
	\begin{center}
		\includegraphics[width=0.95\textwidth]{Chapter_6/fullfield_100_0}
	\end{center}
	\caption{The top surface out-of-plane velocity snapshots for (\textbf{a}) the \acf{fcgm}, (\textbf{b}) the experimental results obtained by the \acf{sldv}, and (\textbf{c}) the \acf{hcgm} in the \textbf{healthy~sample at 100 kHz}}
	\label{fig:fullfield_100_0}
\end{figure}
\begin{figure}[!hbt]
	\begin{center}
		\includegraphics[width=0.95\textwidth]{Chapter_6/fullfield_150_0}
	\end{center}
	\caption{The top surface out-of-plane velocity snapshots for (\textbf{a}) the \acf{fcgm}, (\textbf{b}) the experimental results obtained by the \acf{sldv}, and (\textbf{c}) the \acf{hcgm} in the \textbf{healthy~sample at 150 kHz}}
	\label{fig:fullfield_150_0}
\end{figure}

In case of the damaged sample, the wavefront was not distorted in the damage area bounded by dashed white rectangle in Figures~\ref{fig:fullfield_50_5}, \ref{fig:fullfield_100_5} and \ref{fig:fullfield_150_5} for all three cases.
Due to the lack of wave leakage into the core, the wave propagated smoothly through the skin.
For the experimental sample and the \ac{fcgm}, interference of waves reflected from the cells and the damage boundary is observed.
The wave interference observed in the \ac{hcgm} refers to waves reflected only from the defect.

\begin{figure}[!hbt]
	\begin{center}
		\includegraphics[width=0.95\textwidth]{Chapter_6/fullfield_50_5}
	\end{center}
	\caption{The top surface out-of-plane particle velocity snapshots for (\textbf{a}) the \acf{fcgm}, (\textbf{b}) the experimental results obtained by the \acf{sldv}, and (\textbf{c}) the \acf{hcgm} \textbf{with removed core elements at 50 kHz} in damaged area for both numerical models}
	\label{fig:fullfield_50_5}
\end{figure}
\begin{figure}[!hbt]
	\begin{center}
		\includegraphics[width=0.95\textwidth]{Chapter_6/fullfield_100_5}
	\end{center}
	\caption{The top surface out-of-plane velocity snapshots for (\textbf{a}) the \acf{fcgm}, (\textbf{b}) the experimental results obtained by the \acf{sldv}, and (\textbf{c}) the \acf{hcgm} \textbf{with removed core elements at 100 kHz} in damaged area for both numerical models}
	\label{fig:fullfield_100_5}
\end{figure}
\begin{figure}[!hbt]
	\begin{center}
		\includegraphics[width=0.95\textwidth]{Chapter_6/fullfield_150_5}
	\end{center}
	\caption{The top surface out-of-plane velocity snapshots for (\textbf{a}) the \acf{fcgm}, (\textbf{b}) the experimental results obtained by the \acf{sldv}, and (\textbf{c}) the \acf{hcgm} \textbf{with removed core elements at 150 kHz} in damaged area for both numerical models}
	\label{fig:fullfield_150_5}
\end{figure}
\clearpage