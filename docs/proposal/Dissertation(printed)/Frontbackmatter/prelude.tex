% prelude.tex (specification of which features in `mathphdthesis.sty' you
% are using, your personal information, and your title & abstract)

% Specify features of `mathphdthesis.sty' you want to use:
\titlepgtrue 												% Main title page (required)
\signaturepagetrue 											% Page for declaration of originality (required)
\copyrighttrue 												% Copyright page (required)
\abswithesistrue 											% Abstract to be bound with thesis (optional)
\acktrue 													% Acknowledgments page (optional)
\tablecontentstrue 											% Table of contents page (required)
\tablespagetrue 											% Table of contents page for tables (required only if you have tables)
\figurespagetrue 											% Table of contents page for figures (required only if you have figures)

% Title, author, supervisors, university, date of submission
% Thesis title
\author{\textit{Piotr Fiborek}} 	% First name and surname of candidate (e.g. John Doe)
\prevdegrees{\textit{M.Sc. Eng.}}			% Specify your previous degrees (e.g. B.E. (Hons))
\institute{Mechanics of Intelligent Structures Department}								% Institute of department (e.g. National Centre for Maritime Engineering and Hydrodynamics)

\title{\textbf{Modelling of sandwich plates and piezoelectric transducers to identify \\the severity of mechanical damage}}		

\submittedfor{A dissertation submitted to the Scientific Board of Institute of Fluid Flow Machinery, Polish Academy of Sciences in partial fulfillment of the requirements\\ for the Degree of Doctor of Philosophy}			% Degree thesis is submitted for (e.g. Submitted in fulfillment of the requirements for the Degree of Doctor of Philosophy)
\advisor{\textit{Pawe\l{} Kudela, D.Sc. Ph.D. Eng.}} % Supervisors:
\dept{Institute of Fluid Flow Machinery, Polish Academy of Sciences}
\submitdate{Gdańsk, \textit{February, 2023}}						% Month & year of your thesis submission (e.g. January, 2016)

% Abstract to be bound with thesis
\newcommand{\abstextwithesis}
{
The dissertation was aimed at developing a model-assisted method for assessing the damage in a sandwich panel with a honeycomb core. For this purpose, a structural health monitoring technique based on guided wave propagation was employed. The research offered new insight into modelling the propagation of elastic waves in a complex structure to determine how damage size affects the characteristic propagation parameters.

A function of the effect of damage size on wave propagation could be obtained by experimental investigation or theoretical analysis, but both methods are subject to certain limitations. Experimental investigation would require many expensive samples. Theoretical analysis, on the other hand, is restricted to fundamental structural elements (such as plates, bars and beams) under specific boundary conditions. In contrast, numerical modelling provides accurate data and can be flexibly adapted to different engineering structures without straining the research budget.

The numerical analysis of guided wave propagation is very time-consuming and operationally memory-intensive. Therefore, the research employed the time-domain spectral element method, one of the most accurate and efficient techniques for modelling wave propagation. The time integration algorithm was vectorised to conduct parallel computation on a multi-core graphics card. Also, computational efficiency was improved by reducing global degrees of freedom using two-dimensional elements.
Approaches to determining the material properties of composites are mainly based on the homogenisation process. However, this method is insufficient for modelling wave propagation in materials with complex structures, such as honeycomb sandwich composites. Therefore, the full core geometry model of the material structure was developed in the proposed research.
The honeycomb structure was modelled by shell spectral elements for each cell wall.

The analysis was a multiphysics approach that considered an electromechanical coupling corresponding to using piezoelectric transducers for signal excitation and sensing. The model was developed taking into account the effect of ambient temperature. In addition, to join all of the structure's components, interface elements were implemented to guarantee the continuity of the displacements of adjacent elements. A~new approach was used to develop an interface based on Lagrange multipliers using spectral element shape functions to connect two non-matching grids.

The experimental measurements were conducted for model validation using (i) the impedance analyser for the piezoelectric transducers model, (ii) the scanning Doppler laser vibrometer for full-field wave propagation analysis and (iii) the piezoelectric wave acquisition setup for spot analysis of the wave propagation in the structure.

The literature review on the subject was presented at the beginning of the dissertation, alongside the description of the methodology adopted to achieve the stated purpose of the work. Then, a detailed description of the model implementation of wave propagation in a honeycomb sandwich structure using the spectral element method was presented.
Once the model had been validated, a numerical analysis was provided to determine the model-assisted damage identification function considering the varying ambient temperature. Then, parametric studies were carried out to demonstrate the effect of structural parameters on wave propagation in a honeycomb core structure. Lastly, the dissertation was summarised with conclusions arising from the analyses. 

The novelty of the research was that it developed a new approach to assessing the severity of damage and a numerical model with accurate geometry of a honeycomb structure that uses the spectral element method under varying ambient temperatures. The method that was developed could be a practical compendium of knowledge for designers of structural health monitoring.
In addition, a non-matching interface was developed to connect the two components in the frame of the spectral element method, making this technique more flexible when modelling complex structures.
}
% Abstract to be bound with thesis
\newcommand{\strtextwithesis}
{
Celem pracy było opracowanie wspomaganej modelem metody oceny uszko\-dzeń mechanicznych w~płycie warstwowej z~rdzeniem o~strukturze plastra miodu. W~tym~celu zastosowano technikę monitorowania stanu technicznego konstrukcji opartą na propagacji fal prowadzonych. Badania umożliwiły nowe spojrzenie na~modelowanie propagacji fal w~złożonej strukturze w celu określenia, jak~ro\-zmiar uszkodzenia wpływa na~charakterystyczne parametry propagacji.

Funkcja wpływu wielkości uszkodzenia na propagację fali mogłaby być uzyska\-na poprzez badania eksperymentalne lub analizę teoretyczną, ale obie metody pod\-legają pewnym ograniczeniom. Badania eksperymentalne wymagałyby wielu ko\-sztownych próbek. Analiza teoretyczna, z drugiej strony, jest ograniczona do pod\-stawowych ele\-mentów konstrukcyjnych (takich jak płyty, pręty i belki) w określonych warunkach brzegowych. Natomiast narzędzia modelowania numerycznego dostarczają dokładnych danych, oraz znajdują zastosowanie w~złożonych strukturach inżynierskich bez~nadwyrężania budżetu badań.

Analiza numeryczna propagacji fali kierowanej jest bardzo czasochłonna i~wy\-maga znacznych zasobów pamięci operacyjnej. Dlatego zastosowano metodę ele\-mentów spektralnych w dziedzinie czasu, która jest jedną z najdokładniejszych i~najbardziej wydajnych technik modelowania propagacji fal sprężystych. Algorytm całkowania w~czasie został zwektoryzowany w~celu prowadzenia obliczeń równoległych na~wielo\-rdzeniowej karcie graficznej. Zwiększono również efektywność obliczeniową poprzez redukcję globalnych stopni swobody stosując elementy dwuwymiarowe.
Podejścia do wyznaczania właściwości materiałowych kompozytów opierają się głównie na~procesie homogenizacji. Metoda ta~jest jednak niewystarczająca do~modelo\-wania propagacji fal w~materiałach o~złożonej strukturze, takich jak~ko\-mpozyty wa\-rstwowe o~strukturze plastra miodu. Dlatego w~proponowanych badaniach opraco\-wano model struktury materiału o~pełnej geometrii rdzenia. Struktura pla\-stra miodu została zamodelowana za~pomocą powłokowych elementów spektralnych dla~każdej ściany komó\-rki rdzenia.
\pagebreak

W~analizie zastosowano podejście wielofizyczne, w~którym uwzględniono sprzężenie elektromechaniczne odpowiadające zastosowaniu przetworników piezo\-elektrycznych do~wzbudzania i~rejestrowania sygnału. Model został opracowany z~u\-względnieniem wpływu temperatury otoczenia. Dodatkowo, celem połączenia wszystkich kompone\-ntów struktury, zaimplementowano elementy interfejsowe gwarantujące ciągłość przemie\-szczeń sąsiednich elementów. Do~opracowania interfejsu za\-stosowano nowe podejście oparte na~mnożnikach Lagrange'a wykorzystujące funkcje kształtu elementów spektra\-lnych do~po\-łączenia dwóch niepasujących siatek.

W~celu walidacji modelu przeprowadzono pomiary eksperymentalne z wykorzy\-staniem (i)~analizatora impedancji do przetworników piezoelektry\-cznych, (ii) skanują\-cego wibrometru laserowego Dopplera do~analizy pełnego pola propagacji fali oraz~(iii) piezoelektrycznego zestawu akwizycji fal do~punktowej analizy pro\-pagacji fal w~stru\-kturze.

Przegląd literatury przedmiotu został przedstawiony na~początku rozprawy, gdzie opisano również metodologię, która została przyjęta do~realizacji założonego celu pracy. Następnie przedstawiono szczegółowy opis realizacji modelu propagacji fali w~strukturze warstwowej o~strukturze plastra miodu z~wykorzystaniem metody elementów spektralnych.
Po walidacji modelu przedstawiono analizę numeryczną mającą na~celu wyznaczenie funkcji identyfikacji uszkodzeń wspomaganej modelem z~uwzglę\-dnieniem zmiennej temperatury otoczenia. Następnie przeprowadzono badania parametryczne w~celu przedstawienia wpływu współczyników materiałowych płyty z~rdzeniem o~strukturze plastra miodu na~propagację fali. Na~koniec podsumowano rozprawę, przedstawiając wnioski wynikające z~przeprowadzonych analiz. 

Nowatorskość przeprowadzonych badań polegała na~opracowaniu nowego podejścia do~oceny stopnia uszkodzenia oraz~modelu numerycznego z~dokładną geo\-metrią rdzenia o~strukturze plastra miodu wykorzystując metodę elementów spe\-ktralnych w~warunkach zmiennej temperatury otoczenia. Opracowana metoda może stanowić praktyczne kompendium wiedzy dla~projektantów systemów monitorowania stanu technicznego konstrukcji. Ponadto, w~ramach metody elementów spektralnych, opracowano interfejs do~po\-łączenia dwóch sąsiednich elementów o~niepasujących siatkach, co~czyni tę~technikę bardziej elastyczną przy modelowaniu złożonych struktur.
}

% Acknowledgments page
\newcommand{\acknowledgement}
{
}

% Engineering guote page
\newcommand{\engineeringquote}
{
\null\vfill
\begin{quote}
"When you want to know how things really work, \\  \hspace*{2cm} study them when they’re coming apart."\begin{flushright}- William Gibson 
\end{flushright}
\end{quote}
\vfill
}

% Bibliography Title
\renewcommand{\bibname}{Bibliography}
% Bibliography spacing
\setlength\bibitemsep{1.5\itemsep}

% Settings for array package
\newcolumntype{L}[1]{>{\raggedright\let\newline\\\arraybackslash\hspace{0pt}}m{#1}}
\newcolumntype{C}[1]{>{\centering\let\newline\\\arraybackslash\hspace{0pt}}m{#1}}
\newcolumntype{R}[1]{>{\raggedleft\let\newline\\\arraybackslash\hspace{0pt}}m{#1}}


\beforepreface
The realisation of the dissertation would not have been possible without the support and encouragement of many people.
First and foremost, I would like to express my gratitude to my thesis supervisor Paweł Kudela, D.Sc., Ph.D., Eng. for his guidance and perceptive view of my thesis.
I am fortunate to have grown my interests and research career with him.

I would like to thank Wiesław Ostachowicz, Prof., D.Sc., Ph.D., Eng.,
Corresponding Member of the Polish Academy of Sciences, Head of the Centre of Mechanics of Machinery and the Mechanics of Intelligent Structures Department.
It is a great honour to be a member of the team pioneering research in structural health monitoring.
I also thank Alfred Zmitrowicz, D.Sc., Ph.D., Eng., Head of Doctoral Studies at the Institute of Fluid Flow Machinery.
I am also grateful to Maciej Radzieński, D.Sc., Ph.D., Eng. and Tomasz Wandowski, D.Sc., Ph.D., Eng., for sharing enormous knowledge of laser vibrometer and electromechanical impedance measurements,~respectively.

Moreover, I wish to thank all the Mechanics of Intelligent Structures Department researchers who supported me with their knowledge and professional experience during theoretical and experimental work: Grzegorz Zboiński, Prof., D.Sc., Ph.D., Eng., Katarzyna Majewska, D.Sc., Ph.D., Paweł Malinowski, D.Sc., Ph.D., Magdalena Mieloszyk, D.Sc., Ph.D., Eng., Shirsendu Sikdar, Ph.D., Eng. and Rohan Soman, Ph.D., Eng.

I would acknowledge the Polish National Science for financial support under grant agreement no. 2018/31/N/ST8/02865 and Renata Opieka-Sowińska, M.Sc. and Monika Ratajczak, M.Sc. for administrative services of the project.

I wish to thank all my friends and family who encouraged me until the end of my dissertation writing.
Special thanks go to my parents, who passed on what is most important in life.

Last but not least, I would like to acknowledge my beloved girls, my wife \mbox{Honorata}, and my daughters Hanna and Emilia, for being my inspiration and their endless support throughout the dissertation preparation.
\afterpreface